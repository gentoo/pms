\chapter{metadata.xml}
\label{sec:metadata-xml}

The \t{metadata.xml} file is used to contain extra package- or category-level information beyond
what is stored in ebuild metadata. Its exact format is strictly beyond the scope of this document,
and is described in GLEP 68~\cite{Glep68}.

\chapter{Unspecified Items}

The following items are not specified by this document, and must not be relied upon by ebuilds.
This is, of course, an incomplete list---it covers only the things that the authors know have
been abused in the past.

\begin{compactitem}
\item The \t{FEATURES} variable. This is Portage specific.
\item Similarly, any \t{EMERGE_} variable and any \t{PORTAGE_} variable not explicitly listed.
\item Any Portage configuration file.
\item The VDB (\t{/var/db/pkg}). Ebuilds must not access this or rely upon it existing or being
    in any particular format.
\item The \t{portageq} command. The \t{has_version} and \t{best_version} commands are
    available as functions.
\item The \t{emerge} command.
\item Binary packages.
\item The \t{PORTDIR_OVERLAY} variable, and overlay behaviour in general.
\end{compactitem}

\chapter{Historical Curiosities}

\section{Long-obsolete Features}

The items described in this section are included for information only. Unless otherwise noted, they
were deprecated or abandoned long before EAPI was introduced. Ebuilds must not use these features,
and package managers should not be changed to support them.

\subsection{If-else USE blocks}
Historically, Portage supported if-else use conditionals, as shown by
listing~\ref{lst:if-else-use-listing}. The block before the colon would be taken if the condition
was met, and the block after the colon would be taken if the condition was not met.

\begin{listing}
\caption{If-else use blocks} \label{lst:if-else-use-listing}
\begin{verbatim}
DEPEND="
    flag? (
        taken/if-true
    ) : (
        taken/if-false
    )
    "
\end{verbatim}
\end{listing}

\subsection{CVS versions}
Portage has very crude support for CVS packages. The package \t{foo} could contain a file named
\t{foo-cvs.1.2.3.ebuild}. This version would order \e{higher} than any non-CVS version (including
\t{foo-2.ebuild}). This feature has not seen real world use and breaks versioned dependencies, so
it must not be used.

\subsection{use.defaults}
The \t{use.defaults} file in the profile directory was used to implement `autouse'---switching USE
flags on or off depending upon which packages are installed. It was deprecated long ago and finally
removed in 2009.

\section{Retroactive Changes}

In some exceptional cases, changes to the specification have been approved by the Gentoo Council
without introducing a new EAPI. This section lists such retroactive changes.

\subsection{Bash version}
EAPIs \t{0}, \t{1} and \t{2} originally specified GNU Bash version 3.0. This was retroactively
updated to version 3.2 (see table~\ref{tab:bash-version}) in November 2009.

\subsection{Old-style virtuals}
Historically, virtuals were special packages rather than regular ebuilds. An ebuild could specify in
the \t{PROVIDE} metadata that it supplied certain virtuals, and the package manager had to bear this
in mind when handling dependencies.

Old-style virtuals were supported by EAPIs \t{0}, \t{1}, \t{2}, \t{3} and \t{4}. They were phased
out via GLEP 37~\cite{Glep37} and finally removed in 2011.

\subsection{EAPI parsing}
The method to specify the EAPI of an ebuild used to be a shell variable assignment, and the
package manager had to source the ebuild in order to determine the EAPI. Therefore any ebuild using
a future EAPI would still have to be sourceable by old package managers, which imposed restrictions
e.\,g.\ on updating the Bash version or on possible changes of global scope functions. Several
approaches to overcome this limitation were discussed, notably GLEP 55~\cite{Glep55}, which was
rejected though.

The current syntax of the \t{EAPI} assignment statement (see section~\ref{sec:eapi}), allowing
the package manager to obtain the EAPI from the ebuild by a regular expression match and without
sourcing it, was introduced in May 2012.

\subsection{Package names}
Previously, package names were only required not to end in a hyphen followed by one or more digits.
In October 2012 this was tightened to the specification in section~\ref{sec:package-names}, namely
that they must not end in a hyphen followed by anything resembling a package version.

\subsection{Asterisk in dependency specification}
In the \t{=} dependency operator specified in section~\ref{sec:dep-operator}, an asterisk used to
induce string prefix comparison instead of the normal version comparison logic. That could lead to
surprising results, e.\,g.\ \t{=dev-lang/perl-5.2*} matching \t{dev-lang/perl-5.22.0}. Moreover,
implementation in package managers deviated from what was specified.

String prefix matching was effective in EAPIs \t{0}, \t{1}, \t{2}, \t{3}, \t{4} and \t{5}. It was
retroactively dropped in favour of the current behaviour in October 2015.

\subsection{Empty dependency groups}
The dependency specification format (see section~\ref{sec:dependency-spec}) originally permitted
all-of, any-of, exactly-one-of, at-most-one-of and use-conditional groups with zero sub-elements.
However, such empty groups were neither supported by all package managers nor used in ebuilds.
They were dropped from the specification in October 2017.

% vim: set filetype=tex fileencoding=utf8 et tw=100 spell spelllang=en :

%%% Local Variables:
%%% mode: latex
%%% TeX-master: "pms"
%%% LaTeX-indent-level: 4
%%% LaTeX-item-indent: 0
%%% TeX-brace-indent-level: 4
%%% fill-column: 100
%%% End:
