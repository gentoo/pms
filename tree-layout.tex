\chapter{Tree Layout}

\section{Top Level}

An ebuild repository shall occupy one directory on disk, with the following subdirectories:
\begin{itemize}
\item One directory per category, whose name shall be the name of the category. The layout of
    these directories shall be as described in section \ref{category-dirs}.
\item A \t{profiles} directory (optional), described in section \ref{profiles-dir}.
\item A \t{licenses} directory (optional), described in section \ref{licenses-dir}.
\item An \t{eclass} directory (optional), described in section \ref{eclass-dir}.
\item A \t{metadata} directory (optional), described in section \ref{metadata-dir}.
\item Other optional support files (skeleton ebuilds or ChangeLogs, for example) may exist
    but are not covered by this specification.

\end{itemize}

\section{Category Directories}
\label{category-dirs}

Each category provided by the repository (see also: the \t{profiles/categories} file, section
\ref{profiles-categories}) shall be contained in one directory, whose name shall be that of the
category. Each category directory shall contain:
\begin{itemize}
\item A \t{metadata.xml} file, as described in appendix \ref{metadata-xml}. Optional.
\item Zero or more package directories, one for each package in the category, as described in section
    \ref{package-dirs}. The name of the package directory shall be the corresponding package name.
\end{itemize}

\section{Package Directories}
\label{package-dirs}

A package directory contains the following:
\begin{itemize}
\item One or more ebuilds. These are as described in section \ref{ebuild-format} and others.
\item A \t{metadata.xml} file, as described in appendix \ref{metadata-xml}.
\item A \t{ChangeLog}, in a format determined by the provider of the respository. Optional.
\item A \t{Manifest} file, whose format is described in \cite{Glep44}
\item A \t{files} directory, containing any support files needed by the ebuilds.
\end{itemize}

\section{The Profiles Directory}
\label{profiles-dir}

The profiles directory shall contain zero or more profile directories as described in section
\ref{profiles}, as well as the following files and directories. In any line-based file, lines
beginning with a \# character are treated as comments, while blank lines are ignored.
\begin{description}
\item[arch.list] Contains a list, one entry per line, of permissible values for the \t{ARCH}
    variable.
\item[categories] Contains a list, one entry per line, of categories provided by this repository.
\item[info\_pkgs] Contains a list, one entry per line, of (EAPI-0) dependency atoms. Any package
    matching one of these is to be listed when a package manager displays a `system information'
    listing.
\item[info\_vars] Contains a list, one entry per line, of profile, configuration, and environment
    variables which are considered to be of interest. The value of each of these is to be included
    when the package manager displays a `system information' listing.
\item[package.mask] Contains a list, one entry per line, of (EAPI-0) dependency atoms. Any package
    version matching one of these is considered to be masked, and will not be installed regardless
    of profile unless it is unmasked by the user configuration.
\item[profiles.desc] Described below in section \ref{profiles.desc}.
\item[repo\_name] Contains, on a single line, the name of this repository. The repository name must
    conform to section \ref{repository-names}.
\item[thirdpartymirrors] Described below in section \ref{thirdpartymirrors}.
\item[use.desc] Contains descriptions of valid global USE flags for this repository. The format is
    described in section \ref{use.desc}.
\item[use.local.desc] Contains descriptions of valid local USE flags for this repository, along with
    the packages to which they apply. The format is as described in section \ref{use.desc}.
\item[desc/] This directory contains files analogous to \t{use.desc} for the various \t{USE\_EXPAND}
    variables. Each file in it is named \t{<varname>.desc}, where \t{<varname>} is the variable
    name, in lowercase, whose possible values the file describes. The format of each file is as for
    \t{use.desc}, described in section \ref{use.desc}.
\item[updates/] This directory is described in section \ref{updates-dir}.
\end{description}

\subsection{The profiles.desc file}
\label{profiles.desc}

\subsection{The thirdpartymirrors file}
\label{thirdpartymirrors}

\subsection{The use.desc file}
\label{use.desc}

\subsection{The updates directory}
\label{updates-dir}

\section{The Licenses Directory}
\label{licenses-dir}

The \t{licenses} directory shall contain copies of the licenses used by packages in the
repository. Each file will be named according to the name used in the \t{LICENSE} variable as
described in section \ref{ebuild-var-LICENSE}, and will contain the complete text of the license in
human-readable form. Plain text and PDF formats are permitted, with the former strongly preferred.

\section{The Eclass Directory}
\label{eclass-dir}

The \t{eclass} directory shall contain copies of the eclasses provided by this repository. The
format of these files is described in section \ref{eclasses}. It may also contain, in their own
directory, support files needed by these eclasses.

\section{The Metadata Directory}
\label{metadata-dir}

% vim: set filetype=tex fileencoding=utf8 et tw=100 spell spelllang=en :
