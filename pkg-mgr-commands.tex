The following commands will always be available in the ebuild environment, provided by the package
manager:

\subsubsection{Sandbox commands}
These commands affect the behaviour of the sandbox. Each command takes a single directory as
argument.
\begin{description}
\item[addread] Add a directory to the permitted read list.
\item[addwrite] Add a directory to the permitted write list.
\item[addpredict] Add a directory to the predict list. Any write to a location in this list will be
    denied, but will not trigger access violation messages or abort the build process.
\item[adddeny] Add a directory to the deny list.
\end{description}

\subsubsection{Package manager query commands}
These commands are used to extract information about the host system.
\begin{description}
\item[has\_version] Takes exactly one dependency atom as an argument. Returns true if a package
    matching the atom is installed in \t{\$ROOT}, and false otherwise.
\item[best\_version] Takes exactly one dependency atom as an argument. If a matching package is
    installed, prints the category, package name and version of the highest matching version.
\end{description}

\subsubsection{Output commands}
These commands display messages to the user. Unless otherwise stated, the entire argument list is
used as a message, as in the simple invocations of \t{echo}.
\begin{description}
\item[einfo] Displays an informational message.
\item[einfon] Displays an informational message without a trailing newline.
\item[elog] Displays an informational message of slightly higher importance. The package manager may
    choose to log \t{elog} messages by default where \t{einfo} messages are not, for example.
\item[ewarn] Displays a warning message.
\item[eerror] Displays an error message.
\item[ebegin] Displays an informational message. Should be used when beginning a possibly lengthy
    process, and followed by a call to \t{eend}.
\item[eend] Indicates that the process begun with an \t{ebegin} message has completed. Takes one
    fixed argument, which is a numeric return code, and an optional message in all subsequent
    arguments. If the first argument is 0, print a success indicator; otherwise, print the message
    followed by a failure indicator.
\end{description}

\subsubsection{Error commands}
These commands are used when an error is detected that will prevent the build process from
completing.
\begin{description}
\item[die] Displays a failure message provided in its first and only argument, and then aborts the
    build process. \t{die} is \e{not} guaranteed to work correctly if called from a subshell
    environment.
\item[assert] Checks the value of the shell's pipe status variable, and if any component is non-zero
    (indicating failure), calls \t{die} with its first argument as a failure message.
\end{description}

% vim: set filetype=tex fileencoding=utf8 et tw=100 spell spelllang=en :
