\label{sec:pkg-mgr-commands}

The following commands will always be available in the ebuild environment, provided by the package
manager. Except where otherwise noted, they may be internal (shell functions or aliases) or external
commands available in \t{PATH}; where this is not specified, ebuilds may not rely upon either
behaviour.

\subsubsection{Failure behaviour and related commands}
\label{sec:failure-behaviour}

\featurelabel{die-on-failure} Where a command is listed as having EAPI dependent failure behaviour,
a failure shall either result in a non-zero exit status or abort the build process, as determined by
table~\ref{tab:commands-die-table}.

The following commands affect this behaviour:
\begin{description}
\item[nonfatal] \featurelabel{nonfatal} Executes the remainder of its arguments as a command,
    preserving the exit status. If this results in a command being called that would normally abort
    the build process due to a failure (but not due to an explicit \t{die} or \t{assert} call),
    instead a non-zero exit status shall be returned. Only in EAPIs listed in
    table~\ref{tab:commands-die-table} as supporting \t{nonfatal}.
\end{description}

\ChangeWhenAddingAnEAPI{5}
\begin{centertable}{EAPI Command Failure Behaviour} \label{tab:commands-die-table}
    \begin{tabular}{ l l l }
        \toprule
            \multicolumn{1}{c}{\textbf{EAPI}} &
            \multicolumn{1}{c}{\textbf{Command failure behaviour}} &
            \multicolumn{1}{c}{\textbf{Supports \t{nonfatal}?}} \\
            \midrule
    \t{0} & Non-zero exit & No \\
    \t{1} & Non-zero exit & No \\
    \t{2} & Non-zero exit & No \\
    \t{3} & Non-zero exit & No \\
    \t{4} & Aborts & Yes \\
    \t{5} & Aborts & Yes \\
    \bottomrule
    \end{tabular}
\end{centertable}

\subsubsection{Banned commands}
\label{sec:banned-commands}

\featurelabel{banned-commands} Some commands are banned in some EAPIs. If a banned command is
called, the package manager must abort the build process indicating an error.

\ChangeWhenAddingAnEAPI{5}
\begin{centertable}{Banned commands} \label{tab:banned-commands-table}
    \begin{tabular}{ l l l l }
        \toprule
        \multicolumn{1}{c}{\textbf{EAPI}} &
        \multicolumn{3}{c}{\textbf{Command banned?}} \\
        \multicolumn{1}{c}{} &
        \multicolumn{1}{c}{\textbf{\t{dohard}}} &
        \multicolumn{1}{c}{\textbf{\t{dosed}}} \\
        \midrule
    \t{0} & No & No \\
    \t{1} & No & No \\
    \t{2} & No & No \\
    \t{3} & No & No \\
    \t{4} & Yes & Yes \\
    \t{5} & Yes & Yes \\
    \bottomrule
    \end{tabular}
\end{centertable}

\subsubsection{Sandbox commands}
These commands affect the behaviour of the sandbox. Each command takes a single directory as
argument. Ebuilds must not run any of these commands once the current phase function has returned.
\begin{description}
\item[addread] Add a directory to the permitted read list.
\item[addwrite] Add a directory to the permitted write list.
\item[addpredict] Add a directory to the predict list. Any write to a location in this list will be
    denied, but will not trigger access violation messages or abort the build process.
\item[adddeny] Add a directory to the deny list.
\end{description}

\subsubsection{Package manager query commands}
These commands are used to extract information about the host system. Ebuilds must not run any of
these commands in parallel with any other package manager command. Ebuilds must not run any of
these commands once the current phase function has returned.
\begin{description}
\item[has\_version] Takes exactly one package dependency specification as an argument. Returns
    true if a package matching the atom is installed in \t{ROOT}, and false otherwise.
\item[best\_version] Takes exactly one package dependency specification as an argument. If a
    matching package is installed, prints the category, package name and version of the highest
    matching version.
\end{description}

\subsubsection{Output commands}
These commands display messages to the user. Unless otherwise stated, the entire argument list is
used as a message, with backslash-escaped characters interpreted as for the \t{echo -e} command of
bash, notably \t{\textbackslash t} for a horizontal tab, \t{\textbackslash n} for a new line, and
\t{\textbackslash\textbackslash} for a literal backslash. Ebuilds must not run any of these commands
once the current phase function has returned. Unless otherwise noted, output may be sent to stdout,
stderr or some other appropriate facility.
\begin{description}
\item[einfo] Displays an informational message.
\item[einfon] Displays an informational message without a trailing newline.
\item[elog] Displays an informational message of slightly higher importance. The package
    manager may choose to log \t{elog} messages by default where \t{einfo} messages are not, for
    example.
\item[ewarn] Displays a warning message. Must not go to stdout.
\item[eerror] Displays an error message. Must not go to stdout.
\item[ebegin] Displays an informational message. Should be used when beginning a possibly
    lengthy process, and followed by a call to \t{eend}.
\item[eend] Indicates that the process begun with an \t{ebegin} message has completed. Takes one
    fixed argument, which is a numeric return code, and an optional message in all subsequent
    arguments. If the first argument is 0, prints a success indicator; otherwise, prints the
    message followed by a failure indicator. Returns its first argument as exit status.
\end{description}

\subsubsection{Error commands}
These commands are used when an error is detected that will prevent the build process from
completing. Ebuilds must not run any of these commands once the current phase function has returned.
\begin{description}
\item[die] Displays a failure message provided in its first and only argument, and then aborts the
    build process. \t{die} is \e{not} guaranteed to work correctly if called from a subshell
    environment.
\item[assert] Checks the value of the shell's pipe status variable, and if any component is non-zero
    (indicating failure), calls \t{die} with its first argument as a failure message.
\end{description}

\subsubsection{Build commands}
These commands are used during the \t{src\_compile} and \t{src\_install} phases to run the
package's build commands. Ebuilds must not run any of these commands once the current phase function
has returned.

\begin{description}
\item[econf] Calls the program's \t{./configure} script. This is designed to work with GNU
    Autoconf-generated scripts. Any additional parameters passed to \t{econf} are passed directly
    to \t{./configure}. \t{econf} will look in the current working directory for a configure script
    unless the \t{ECONF\_SOURCE} environment variable is set, in which case it is taken to be the
    directory containing it. \t{econf} must pass the following options to the configure script:

    \featurelabel{econf-options}
    \begin{itemize}
    \item -{}-prefix must default to \t{\$\{EPREFIX\}/usr} unless overridden by \t{econf}'s caller.
    \item -{}-mandir must be \t{\$\{EPREFIX\}/usr/share/man}
    \item -{}-infodir must be \t{\$\{EPREFIX\}/usr/share/info}
    \item -{}-datadir must be \t{\$\{EPREFIX\}/usr/share}
    \item -{}-sysconfdir must be \t{\$\{EPREFIX\}/etc}
    \item -{}-localstatedir must be \t{\$\{EPREFIX\}/var/lib}
    \item -{}-host must be the value of the \t{CHOST} environment variable.
    \item -{}-libdir must be set according to Algorithm~\ref{alg:econf-libdir}.
    \item -{}-disable-dependency-tracking, if the EAPI is listed in
        table~\ref{tab:econf-options-table} as using it. This option will only be passed if the
        string \t{disable-dependency-tracking} occurs in the output of \t{configure -{}-help}.
    \item -{}-disable-silent-rules, if the EAPI is listed in
        table~\ref{tab:econf-options-table} as using it. This option will only be passed if the
        string \t{disable-silent-rules} occurs in the output of \t{configure -{}-help}.
    \end{itemize}

    \ChangeWhenAddingAnEAPI{5}
    \begin{centertable}{Extra \t{econf} arguments for EAPIs} \label{tab:econf-options-table}
        \begin{tabular}{ l l l l }
            \toprule
                \multicolumn{1}{c}{\textbf{EAPI}} &
                \multicolumn{1}{c}{\textbf{-{}-disable-dependency-tracking}?} &
                \multicolumn{1}{c}{\textbf{-{}-disable-silent-rules}?} \\
                \midrule
        \t{0} & No & No \\
        \t{1} & No & No \\
        \t{2} & No & No \\
        \t{3} & No & No \\
        \t{4} & Yes & No \\
        \t{5} & Yes & Yes \\
        \bottomrule
        \end{tabular}
    \end{centertable}

    Note that the \t{\$\{EPREFIX\}} component represents the same offset-prefix as described in
    Table~\ref{tab:defined_vars}.  It facilitates offset-prefix installations which is supported by
    EAPIs listed in Table~\ref{tab:offset-env-vars-table}. When no offset-prefix installation is in
    effect, \t{EPREFIX} becomes the empty string, making the behaviour of \t{econf} equal for both
    offset-prefix supporting and agnostic EAPIs.

    \t{econf} must be implemented internally---that is, as a bash function and not an external
    script. Should any portion of it fail, it must abort the build using \t{die}, unless run using
    \t{nonfatal}, in which case it must return non-zero exit status.

\begin{algorithm}
\caption{econf -{}-libdir logic} \label{alg:econf-libdir}
\begin{algorithmic}[1]
\STATE let prefix=\$\{EPREFIX\}/usr
\IF{the caller specified -{}-prefix=\$p}
    \STATE let prefix=\$p
\ENDIF
\STATE let libdir=
\IF{the ABI environment variable is set}
    \STATE let libvar=LIBDIR\_\$ABI
    \IF{the environment variable named by libvar is set}
        \STATE let libdir=the value of the variable named by libvar
    \ENDIF
\ENDIF
\IF{libdir is non-empty}
    \STATE pass -{}-libdir=\$prefix/\$libdir to configure
\ENDIF
\end{algorithmic}
\end{algorithm}

\item[emake] Calls the \t{\$MAKE} program, or GNU make if the \t{MAKE} variable is unset.
    Any arguments given are passed directly to the make command, as are the user's chosen
    \t{MAKEOPTS}\@. Arguments given to \t{emake} override user configuration. See also
    section~\ref{sec:guaranteed-system-commands}. \t{emake} must be an external program and cannot
    be a function or alias---it must be callable from e.\,g.\ \t{xargs}. Failure behaviour is EAPI
    dependent as per section~\ref{sec:failure-behaviour}.
\item[einstall] A shortcut for the command given in Listing~\ref{lst:einstall}. Any arguments given
    to \t{einstall} are passed verbatim to \t{emake}, as shown. Failure behaviour is EAPI dependent
    as per section~\ref{sec:failure-behaviour}.

    The variable \t{ED} is defined as in Table~\ref{tab:defined_vars} and depends on the use of an
    offset-prefix. When such offset-prefix is absent, \t{ED} is equivalent to \t{D}. \t{ED} is always
    available in EAPIs that support offset-prefix installations as listed in
    Table~\ref{tab:offset-env-vars-table}, hence EAPIs lacking offset-prefix support should use \t{D}
    instead of \t{ED} in the command given in Listing~\ref{lst:einstall}.
    Variable \t{libdir} is an auxiliary local variable whose value is determined by
    Algorithm~\ref{alg:ebuild-libdir}.
\begin{listing}[H]
  \caption{einstall command}\label{lst:einstall}
  \begin{verbatim}
emake \
   prefix="${ED}"/usr \
   datadir="${ED}"/usr/share \
   mandir="${ED}"/usr/share/man \
   infodir="${ED}"/usr/share/info \
   libdir="${ED}"/usr/${libdir} \
   localstatedir="${ED}"/var/lib \
   sysconfdir="${ED}"/etc \
   -j1 \
   "$@" \
   install
  \end{verbatim}
\end{listing}

\end{description}

\subsubsection{Installation commands}
These commands are used to install files into the staging area, in cases where the package's \t{make
install} target cannot be used or does not install all needed files. Except where otherwise stated,
all filenames created or modified are relative to the staging directory including the offset-prefix
\t{ED} in offset-prefix aware EAPIs, or just the staging directory \t{D} in offset-prefix agnostic
EAPIs. These commands must all be external programs and not bash functions or aliases---that is, they
must be callable from \t{xargs}. Ebuilds must not run any of these commands once the current phase
function has returned.

\begin{description}
\item[dobin] Installs the given files into \t{DESTTREE/bin}, where
    \t{DESTTREE} defaults to \t{/usr}. Gives the files mode \t{0755}
    and transfers file ownership to the superuser or its equivalent on
    the system or installation at hand. For instance on Gentoo Linux in
    a non-offset-prefix installation this ownership is \t{root:root},
    while on an offset-prefix aware installation this may be
    \t{joe:users}. Failure behaviour is EAPI dependent as per
    section~\ref{sec:failure-behaviour}.

\item[doconfd] Installs the given config files into \t{/etc/conf.d/}, by default with file mode
    \t{0644}. This can be overridden by setting \t{INSOPTIONS} with the \t{insopts} function.
    Failure behaviour is EAPI dependent as per section~\ref{sec:failure-behaviour}.

\item[dodir] Creates the given directories, by default with file mode \t{0755}. This can be overridden
    by setting \t{DIROPTIONS} with the \t{diropts} function. Failure behaviour is EAPI dependent as
    per section~\ref{sec:failure-behaviour}.

\item[dodoc] \featurelabel{dodoc} Installs the given files into a subdirectory under
    \t{/usr/share/doc/\$\{PF\}/} with file mode \t{0644}. The subdirectory is set by the most recent
    call to \t{docinto}. If \t{docinto} has not yet been called, instead installs to the directory
    \t{/usr/share/doc/\$\{PF\}/}. For EAPIs listed in table~\ref{tab:dodoc-table} as supporting \t{-r},
    if the first argument is \t{-r}, any subsequent arguments that are directories are installed
    recursively to the appropriate location; in any other case, it is an error for a directory to be
    specified. Failure behaviour is EAPI dependent as per section~\ref{sec:failure-behaviour}.

\item[doenvd] Installs the given environment files into \t{/etc/env.d/}, by default with file mode
    \t{0644}. This can be overridden by setting \t{INSOPTIONS} with the \t{insopts} function.
    Failure behaviour is EAPI dependent as per section~\ref{sec:failure-behaviour}.

\item[doexe] Installs the given files into the directory specified by the most recent \t{exeinto}
    call, by default with file mode \t{0755}. This can be overridden by setting \t{EXEOPTIONS} with
    the \t{exeopts} function. If \t{exeinto} has not yet been called, behaviour is undefined.
    Failure behaviour is EAPI dependent as per section~\ref{sec:failure-behaviour}.

\item[dohard] Takes two parameters. Creates a hardlink from the second to the first. In EAPIs
    listed in table~\ref{tab:banned-commands-table}, this command is banned as per
    section~\ref{sec:banned-commands}.

\item[dohtml] Installs the given HTML files into a subdirectory under \t{/usr/share/doc/\$PF/}.
The subdirectory is \t{html} by default, but this can be overridden by setting the \t{DOCDESTTREE}
variable with the \t{docinto} function. Files to be installed automatically are determined by
extension and the default extensions are \t{css}, \t{gif}, \t{htm}, \t{html}, \t{jpeg}, \t{jpg},
\t{js} and \t{png}. These default extensions can be extended or reduced (see below). The options
that can be passed to \t{dohtml} are as follows:
\begin{description}
    \item{\t{-r}} --- enables recursion into directories.
    \item{\t{-V}} --- enables verbosity.
    \item{\t{-A}} --- adds file type extensions to the default list.
    \item{\t{-a}} --- sets file type extensions to only those specified.
    \item{\t{-f}} --- list of files that are able to be installed.
    \item{\t{-x}} --- list of directories that files will not be installed from (only used in
    conjunction with \t{-r}).
    \item{\t{-p}} --- sets a document prefix for installed files, not to be confused with the global
    offset-prefix.
\end{description}

    Failure behaviour is EAPI dependent as per section~\ref{sec:failure-behaviour}.

    It is undefined whether a failure shall occur if \t{-r} is not specified and a directory is
    encountered. Ebuilds must not rely upon any particular behaviour.

\item[doinfo] Installs a GNU Info file into the \t{/usr/share/info} area with file mode \t{0644}.
    Failure behaviour is EAPI dependent as per section~\ref{sec:failure-behaviour}.

\item[doinitd] Installs the given initscript files into \t{/etc/init.d}, by default with file mode
    \t{0755}. This can be overridden by setting \t{EXEOPTIONS} with the \t{exeopts} function.
    Failure behaviour is EAPI dependent as per section~\ref{sec:failure-behaviour}.

\item[doins] \featurelabel{doins} Takes any number of files as arguments and installs them into
    \t{INSDESTTREE}, by default with file mode \t{0644}. This can be overridden by setting
    \t{INSOPTIONS} with the \t{insopts} function. If the first argument is \t{-r}, then operates
    recursively, descending into any directories given. For EAPIs listed in
    table~\ref{tab:doins-table}, \t{doins} must install symlinks as symlinks;
    for other EAPIs, behaviour is undefined if any symlink is encountered. Failure
    behaviour is EAPI dependent as per section~\ref{sec:failure-behaviour}.

\item[dolib] For each argument, installs it into the appropriate library subdirectory under
    \t{DESTTREE}, as determined by Algorithm~\ref{alg:ebuild-libdir}. The file mode is \t{0644}
    by default. This can be overridden by setting \t{LIBOPTIONS} with the \t{libopts} function.
    Any symlinks are installed into the same directory as relative links to their original target.
    Failure behaviour is EAPI dependent as per section~\ref{sec:failure-behaviour}.

\item[dolib.so] As for dolib except each file is installed with mode \t{0755}.

\item[dolib.a] As for dolib except each file is installed with mode \t{0644}.

\begin{algorithm}
\caption{Determining the library directory} \label{alg:ebuild-libdir}
\begin{algorithmic}[1]
\IF{CONF\_LIBDIR\_OVERRIDE is set in the environment}
    \STATE return CONF\_LIBDIR\_OVERRIDE
\ENDIF
\IF{CONF\_LIBDIR is set in the environment}
    \STATE let LIBDIR\_default=CONF\_LIBDIR
\ELSE
    \STATE let LIBDIR\_default=``lib''
\ENDIF
\IF{ABI is set in the environment}
    \STATE let abi=ABI
\ELSIF{DEFAULT\_ABI is set in the environment}
    \STATE let abi=DEFAULT\_ABI
\ELSE
    \STATE let abi=``default''
\ENDIF
\STATE return the value of LIBDIR\_\$abi
\end{algorithmic}
\end{algorithm}

\item[doman] Installs a man page into the appropriate subdirectory of \t{/usr/share/man} depending
    upon its apparent section suffix (e.\,g.\ \t{foo.1} goes to \t{/usr/share/man/man1/foo.1} with
    file mode \t{0644}.

    \featurelabel{doman-langs} In EAPIs listed in table~\ref{tab:doman-table} as supporting
    language detection by filename, a man page with name of the form \t{foo.}\i{lang}\t{.1} shall
    go to \t{/usr/share/man/}\i{lang}\t{/man1/foo.1}, where \i{lang} refers to a pair of lower-case
    ASCII letters optionally followed by an underscore and a pair of upper-case ASCII letters.
    Failure behaviour is EAPI dependent as per section~\ref{sec:failure-behaviour}.

    With option \t{-i18n=}\i{lang}, a man page shall be installed into an appropriate subdirectory
    of \t{/usr/share/man/}\i{lang} (e.\,g.\ \t{/usr/share/man/}\i{lang}\t{/man1/foo.pl.1} would be
    the destination for \t{foo.pl.1}). The \i{lang} subdirectory level is skipped if \i{lang} is
    the empty string. In EAPIs specified by table~\ref{tab:doman-table}, the \t{-i18n} option takes
    precedence over the language code in the filename.

\item[domo] Installs a \t{.mo} file with file mode \t{0644} into the appropriate subdirectory of
    \t{DESTTREE/share/locale}, generated by taking the basename of the file, removing the \t{.*}
    suffix, and appending \t{/LC\_MESSAGES}. The name of the installed files is the package name
    with \t{.mo} appended. Failure behaviour is EAPI dependent as per section~\ref{sec:failure-behaviour}.

\item[dosbin] As \t{dobin}, but installs to \t{DESTTREE/sbin}.

\item[dosym] Creates a symbolic link named as for its second parameter, pointing to the first. If
    the directory containing the new link does not exist, creates it. Failure behaviour is EAPI
    dependent as per section~\ref{sec:failure-behaviour}.

\item[fowners] Acts as for \t{chown}, but takes paths relative to the image directory. Failure
    behaviour is EAPI dependent as per section~\ref{sec:failure-behaviour}.

\item[fperms] Acts as for \t{chmod}, but takes paths relative to the image directory. Failure
    behaviour is EAPI dependent as per section~\ref{sec:failure-behaviour}.

\item[newbin] As for \t{dobin}, but takes two parameters. The first is the file to install; the
    second is the new filename under which it will be installed.

\item[newconfd] As for \t{doconfd}, but takes two parameters as for \t{newbin}.

\item[newdoc] As above, for \t{dodoc}.

\item[newenvd] As above, for \t{doenvd}.

\item[newexe] As above, for \t{doexe}.

\item[newinitd] As above, for \t{doinitd}.

\item[newins] As above, for \t{doins}.

\item[newlib.a] As above, for \t{dolib.a}.

\item[newlib.so] As above, for \t{dolib.so}.

\item[newman] As above, for \t{doman}.

\item[newsbin] As above, for \t{dosbin}.

\item[keepdir] Creates a directory as for \t{dodir}, and an empty file whose name starts with
    \t{.keep} in that directory to ensure that the directory does not get removed by the
    package manager should it be empty at any point. Failure behaviour is EAPI dependent as per
    section~\ref{sec:failure-behaviour}.

\end{description}

\ChangeWhenAddingAnEAPI{5}
\begin{centertable}{EAPIs supporting \t{dodoc -r}} \label{tab:dodoc-table}
    \begin{tabular}{ l l }
        \toprule
            \multicolumn{1}{c}{\textbf{EAPI}} &
            \multicolumn{1}{c}{\textbf{Supports \t{dodoc -r}?}} \\
            \midrule
    \t{0} & No \\
    \t{1} & No \\
    \t{2} & No \\
    \t{3} & No \\
    \t{4} & Yes \\
    \t{5} & Yes \\
    \bottomrule
    \end{tabular}
\end{centertable}

\ChangeWhenAddingAnEAPI{5}
\begin{centertable}{EAPIs supporting symlinks for \t{doins}} \label{tab:doins-table}
    \begin{tabular}{ l l }
        \toprule
            \multicolumn{1}{c}{\textbf{EAPI}} &
            \multicolumn{1}{c}{\textbf{\t{doins} supports symlinks?}} \\
            \midrule
    \t{0} & No \\
    \t{1} & No \\
    \t{2} & No \\
    \t{3} & No \\
    \t{4} & Yes \\
    \t{5} & Yes \\
    \bottomrule
    \end{tabular}
\end{centertable}

\ChangeWhenAddingAnEAPI{5}
\begin{centertable}{\t{doman} language support options for EAPIs}
    \label{tab:doman-table}
    \begin{tabular}{ l l l }
        \toprule
            \multicolumn{1}{c}{\textbf{EAPI}} &
            \multicolumn{1}{c}{\textbf{Language detection by filename?}} &
            \multicolumn{1}{c}{\textbf{Option \t{-i18n} takes precedence?}} \\
            \midrule
    \t{0} & No & Not applicable \\
    \t{1} & No & Not applicable \\
    \t{2} & Yes & No \\
    \t{3} & Yes & No \\
    \t{4} & Yes & Yes \\
    \t{5} & Yes & Yes \\
    \bottomrule
    \end{tabular}
\end{centertable}

\subsubsection{Commands affecting install destinations}
The following commands are used to set the various destination trees, all relative to \t{\$\{ED\}} in
offset-prefix aware EAPIs and relative to \t{\$\{D\}} in offset-prefix agnostic EAPIs, used by the
above installation commands. They must be shell functions or aliases, due to the need to set variables
read by the above commands. Ebuilds must not run any of these commands once the current phase function
has returned.

\begin{description}

\item[into] Sets the value of \t{DESTTREE} for future invocations
    of the above utilities. Creates the directory under \t{\$\{ED\}}
    in offset-prefix aware EAPIs or under \t{\$\{D\}} in offset-prefix
    agnostic EAPIs, using \t{install -d} with no additional options,
    if it does not already exist. Failure behaviour is EAPI dependent
    as per section~\ref{sec:failure-behaviour}.

\item[insinto] Sets the value of \t{INSDESTTREE} for future invocations of the above utilities. May
    create the directory, as specified for \t{into}.

\item[exeinto] Sets the install path for \t{doexe} and \t{newexe}. May create the directory, as specified
    for \t{into}.

\item[docinto] Sets the install subdirectory for \t{dodoc} et al. May create the directory, as specified
    for \t{into}.

\item[insopts] Sets the options passed by \t{doins} et al. to the \t{install} command.

\item[diropts] Sets the options passed by \t{dodir} et al. to the \t{install} command.

\item[exeopts] Sets the options passed by \t{doexe} et al. to the \t{install} command.

\item[libopts] Sets the options passed by \t{dolib} et al. to the \t{install} command.

\end{description}

\subsubsection{Commands affecting install compression}

\featurelabel{controllable-compress} In EAPIs listed in table~\ref{tab:compression-table} as
supporting controllable compression, the package manager may optionally compress a subset of the files
under the \t{ED} directory in offset-prefix aware EAPIs or the \t{D} directory in offset-prefix
agnostic EAPIs. To control which directories may or may not be compressed, the package manager shall
maintain two lists:

\begin{compactitem}
\item An inclusion list, which initially contains \t{/usr/share/doc}, \t{/usr/share/info} and
    \t{/usr/share/man}.
\item An exclusion list, which initially contains \t{/usr/share/doc/\$\{PF\}/html}.
\end{compactitem}

The optional compression shall be carried out after \t{src\_install} has completed, and before the
execution of any subsequent phase function. For each item in the inclusion list, pretend it has the
value of the \t{ED} variable in offset-prefix aware EAPIs or the \t{D}
variable in offset-prefix agnostic EAPIs prepended, then:

\begin{compactitem}
\item If it is a directory, act as if every file or directory immediately under this directory
    were in the inclusion list.
\item If the item is a file, it may be compressed unless it has been excluded as described below.
\item If the item does not exist, it is ignored.
\end{compactitem}

Whether an item is to be excluded is determined as follows: For each item in the exclusion list,
pretend it has the value of the \t{ED} variable in offset-prefix aware EAPIs or the \t{D} variable in
offset-prefix agnostic EAPIs prepended, then:

\begin{compactitem}
\item If it is a directory, act as if every file or directory immediately under this directory
    were in the exclusion list.
\item If the item is a file, it shall not be compressed.
\item If the item does not exist, it is ignored.
\end{compactitem}

The package manager shall take appropriate steps to ensure that its compression mechanisms behave
sensibly even if an item is listed in the inclusion list multiple times, if an item is a symlink,
or if a file is already compressed.

The following commands may be used in \t{src\_install} to alter these lists. It is an error to call
any of these functions from any other phase.

\begin{description}
\item[docompress] If the first argument is \t{-x}, add each of its subsequent arguments to the
exclusion list. Otherwise, add each argument to the inclusion list. Only available in EAPIs listed
in table~\ref{tab:compression-table} as supporting \t{docompress}.
\end{description}

\ChangeWhenAddingAnEAPI{5}
\begin{centertable}{EAPIs supporting controllable compression} \label{tab:compression-table}
    \begin{tabular}{ l l l }
        \toprule
            \multicolumn{1}{c}{\textbf{EAPI}} &
            \multicolumn{1}{c}{\textbf{Supports controllable compression?}} &
            \multicolumn{1}{c}{\textbf{Supports \t{docompress}?}} \\
            \midrule
    \t{0} & No & No \\
    \t{1} & No & No \\
    \t{2} & No & No \\
    \t{3} & No & No \\
    \t{4} & Yes & Yes \\
    \t{5} & Yes & Yes \\
    \bottomrule
    \end{tabular}
\end{centertable}

\subsubsection{Use List Functions}
These functions provide behaviour based upon set or unset use flags. Ebuilds must not run any of
these commands once the current phase function has returned. Ebuilds must not run any of these
functions in global scope.

If any of these functions is called with a flag value that is not included in \t{IUSE\_EFFECTIVE},
either behaviour is undefined or it is an error as decided by table~\ref{tab:use-list-strictness}.

\begin{description}
\item[use] Returns shell true (0) if the first argument (a \t{USE} flag name) is enabled, false
    otherwise.  If the flag name is prefixed with \t{!}, returns true if the flag is disabled, and
    false if it is enabled. It is guaranteed that this command is quiet.
\item[usev] The same as \t{use}, but also prints the flag name if the condition
    is met.
\item[useq] Deprecated synonym for \t{use}.
\item[use\_with] \featurelabel{use-with} Has one-, two-, and three-argument forms. The first
    argument is a USE flag name, the second a \t{configure} option name (\t{\$\{opt\}}), defaulting
    to the same as the first argument if not provided, and the third is a string value
    (\t{\$\{value\}}). For EAPIs listed in table~\ref{tab:use-with-third-arg} as not supporting it,
    an empty third argument is treated as if it weren't provided. If the USE flag is set, outputs
    \t{-{}-with-\$\{opt\}=\$\{value\}} if the third argument was provided, and
    \t{-{}-with-\$\{opt\}} otherwise. If the flag is not set, then it outputs
    \t{-{}-without-\$\{opt\}}.
\item[use\_enable] Works the same as \t{use\_with()}, but outputs \t{-{}-enable-} or \t{-{}-disable-}
instead of \t{-{}-with-} or \t{-{}-without-}.
\end{description}

\ChangeWhenAddingAnEAPI{5}
\begin{centertable}{EAPI Behaviour for Use Queries not in IUSE\_EFFECTIVE} \label{tab:use-list-strictness}
    \begin{tabular}{ l l }
        \toprule
            \multicolumn{1}{c}{\textbf{EAPI}} &
            \multicolumn{1}{c}{\textbf{Behaviour}} \\
            \midrule
    \t{0} & Undefined \\
    \t{1} & Undefined  \\
    \t{2} & Undefined \\
    \t{3} & Undefined \\
    \t{4} & Error \\
    \t{5} & Error \\
    \bottomrule
    \end{tabular}
\end{centertable}

\ChangeWhenAddingAnEAPI{5}
\begin{centertable}{EAPIs supporting empty third argument in \t{use\_with} and \t{use\_enable}}
    \label{tab:use-with-third-arg}
    \begin{tabular}{ l l }
        \toprule
            \multicolumn{1}{c}{\textbf{EAPI}} &
            \multicolumn{1}{c}{\textbf{Supports empty third argument?}} \\
            \midrule
    \t{0} & No \\
    \t{1} & No \\
    \t{2} & No \\
    \t{3} & No \\
    \t{4} & Yes \\
    \t{5} & Yes \\
    \bottomrule
    \end{tabular}
\end{centertable}

\subsubsection{Text List Functions}
These functions check whitespace-separated lists for a particular value.

\begin{description}
\item[has] Returns shell true (0) if the first argument (a word) is found in the list of subsequent
    arguments, false otherwise. Guaranteed quiet.
\item[hasv] The same as \t{has}, but also prints the first argument if found.
\item[hasq] Deprecated synonym for \t{has}.
\end{description}

\subsubsection{Misc Commands}
The following commands are always available in the ebuild environment, but don't really fit in any
of the above categories. Ebuilds must not run any of these commands once the current phase function
has returned.

\begin{description}
\item[dosed] Takes any number of arguments, which can be files or \t{sed} expressions. For each
    argument, if it names, relative to \t{ED} (offset-prefix aware EAPIs) or \t{D} (offset-prefix agnostic
    EAPIs) a file which exists, then \t{sed} is run with the current expression on that file. Otherwise,
    the current expression is set to the text of the argument. The initial value of the expression is
    \t{s:\$\{ED\}::g} in offset-prefix aware EAPIs and \t{s:\$\{D\}::g} in offset-prefix agnostic
    EAPIs. In EAPIs listed in table~\ref{tab:banned-commands-table}, this command is banned as per
    section~\ref{sec:banned-commands}.

\item[unpack] Unpacks one or more source archives, in order, into the current directory. After
    unpacking, must ensure that all filesystem objects inside the current working directory (but not
    the current working directory itself) have permissions \t{a+r,u+w,go-w} and that all directories
    under the current working directory additionally have permissions \t{a+x}.

    All arguments to \t{unpack} must be either a filename without path, in which case \t{unpack}
    looks in \t{DISTDIR} for the file, or start with the string \t{./}, in which case \t{unpack}
    uses the argument as a path relative to the working directory.

    Any unrecognised file format shall be skipped silently. If unpacking a supported file format
    fails, \t{unpack} shall abort the build process.

    \featurelabel{unpack-extensions} Must be able to unpack the following file formats, if the
    relevant binaries are available:
    \begin{itemize}
    \item tar files (\t{*.tar}). Ebuilds must ensure that GNU tar installed.
    \item gzip-compressed tar files (\t{*.tar.gz, *.tgz, *.tar.Z, *.tbz}). Ebuilds must ensure that
    GNU gzip and GNU tar are installed.
    \item bzip2-compressed tar files (\t{*.tar.bz2, *.tbz2, *.tar.bz}). Ebuilds must ensure that
    bzip2 and GNU tar are installed.
    \item zip files (\t{*.zip, *.ZIP, *.jar}). Ebuilds must ensure that Info-ZIP Unzip is installed.
    \item gzip files (\t{*.gz, *.Z, *.z}). Ebuilds must ensure that GNU gzip is installed.
    \item bzip2 files (\t{*.bz, *.bz2}). Ebuilds must ensure that bzip2 is installed.
    \item 7zip files (\t{*.7z, *.7Z}). Ebuilds must ensure that P7ZIP is installed.
    \item rar files (\t{*.rar, *.RAR}). Ebuilds must ensure that RARLAB's unrar is installed.
    \item LHA archives (\t{*.LHA, *.LHa, *.lha, *.lhz}). Ebuilds must ensure that the lha program is
    installed.
    \item ar archives (\t{*.a}). Ebuilds must ensure that GNU binutils is installed.
    \item deb packages (\t{*.deb}). Ebuilds must ensure that the deb2targz program is installed on
    those platforms where the GNU binutils ar program is not available and the installed ar program is
    incompatible with GNU archives. Otherwise, ebuilds must ensure that GNU binutils is installed.
    \item lzma-compressed files (\t{*.lzma}). Ebuilds must ensure that LZMA Utils is installed.
    \item lzma-compressed tar files (\t{*.tar.lzma}). Ebuilds must ensure that LZMA Utils and
        GNU tar are installed.
    \item xz-compressed files (\t{*.xz}). Ebuilds must ensure that XZ Utils is installed. Only for
        EAPIs listed in table~\ref{tab:unpack-extensions-table} as supporting xz.
    \item xz-compressed tar files (\t{*.tar.xz}). Ebuilds must ensure that XZ Utils and GNU tar are
        installed. Only for EAPIs listed in table~\ref{tab:unpack-extensions-table} as supporting xz.
    \end{itemize}
    It is up to the ebuild to ensure that the relevant external utilities are available, whether by
    being in the system set or via dependencies.

\ChangeWhenAddingAnEAPI{5}
\begin{centertable}{\t{unpack} extensions for EAPIs} \label{tab:unpack-extensions-table}
    \begin{tabular}{ l l }
        \toprule
            \multicolumn{1}{c}{\textbf{EAPI}} &
            \multicolumn{1}{c}{\textbf{\t{.xz} and \t{.tar.xz}?}} \\
            \midrule
    \t{0} & No \\
    \t{1} & No \\
    \t{2} & No \\
    \t{3} & Yes \\
    \t{4} & Yes \\
    \t{5} & Yes \\
    \bottomrule
    \end{tabular}
\end{centertable}

\item[apply\_user\_patches]
    \featurelabel{apply-user-patches} This function is called to indicate to the package manager
    that now would be a suitable time to apply any user patches to the work directory. This function
    must return zero if it is possible that any changes were made to the work directory, and may
    return non-zero if no changes were made. This function must be called at least once in
    \t{src\_prepare}; if the function is called more than once, it is expected that any effects that
    it has shall only be executed once, and that its return value shall only be zero once. Only
    available in EAPIs listed in table~\ref{tab:apply-user-patches-table}.

\ChangeWhenAddingAnEAPI{5}
\begin{centertable}{\t{apply\_user\_patches} support for EAPIs} \label{tab:apply-user-patches-table}
    \begin{tabular}{ l l }
        \toprule
            \multicolumn{1}{c}{\textbf{EAPI}} &
            \multicolumn{1}{c}{\textbf{\t{apply\_user\_patches} support?}} \\
            \midrule
    \t{0} & No \\
    \t{1} & No \\
    \t{2} & No \\
    \t{3} & No \\
    \t{4} & No \\
    \t{5} & Yes \\
    \bottomrule
    \end{tabular}
\end{centertable}

\item[inherit] See section~\ref{sec:inherit}.

\item[default]
    \featurelabel{default-func} Calls the \t{default\_} function for the current phase (see
    section~\ref{sec:default-phase-funcs}).  Must not be called if the \t{default\_} function
    does not exist for the current phase in the current EAPI.  Only available in EAPIs listed in
    table~\ref{tab:default-function-table}.
\end{description}

\ChangeWhenAddingAnEAPI{5}
\begin{centertable}{EAPIs supporting the \t{default} function} \label{tab:default-function-table}
    \begin{tabular}{ l l }
        \toprule
            \multicolumn{1}{c}{\textbf{EAPI}} &
            \multicolumn{1}{c}{\textbf{Supports \t{default} function?}} \\
            \midrule
    \t{0} & No \\
    \t{1} & No \\
    \t{2} & Yes \\
    \t{3} & Yes \\
    \t{4} & Yes \\
    \t{5} & Yes \\
    \bottomrule
    \end{tabular}
\end{centertable}

\subsubsection{Debug Commands}
The following commands are available for debugging. Normally all of these commands should be no ops;
a package manager may provide a special debug mode where these commands instead do something.
Ebuilds must not run any of these commands once the current phase function has returned.

\begin{description}
\item[debug-print] If in a special debug mode, the arguments should be outputted or recorded using
    some kind of debug logging.
\item[debug-print-function] Calls \t{debug-print} with \t{\$1: entering function} as the first
    argument and the remaining arguments as additional arguments.
\item[debug-print-section] Calls \t{debug-print} with \t{now in section \$*}.
\end{description}

\subsubsection{Reserved Commands and Variables}

Except where documented otherwise, all functions and variables that contain any of the following
strings (ignoring case) are reserved for package manager use and may not be used or relied upon by
ebuilds:

\begin{compactitem}
\item \t{abort}
\item \t{dyn}
\item \t{ebuild}
\item \t{hook}
\item \t{paludis}
\item \t{portage}
\item \t{prep}
\end{compactitem}

% vim: set filetype=tex fileencoding=utf8 et tw=100 spell spelllang=en :

%%% Local Variables:
%%% mode: latex
%%% TeX-master: "pms"
%%% LaTeX-indent-level: 4
%%% LaTeX-item-indent: 0
%%% TeX-brace-indent-level: 4
%%% End:
