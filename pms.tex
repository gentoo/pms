\documentclass[a4paper]{report}
% Definition of fonts, choose T1 encoding for fonts
\usepackage[T1]{fontenc}
%
% algorithmic and algorithm to be loaded last to avoid failures
\usepackage{appendix,
  booktabs,
  color,
  enumitem,
  float,
  fullpage,
  graphicx,
  hyperref,
  ifthen,
  longtable,
  paralist,
  parskip,
  verbatim,
  algorithm,
  algorithmic
}
\usepackage[position=top]{caption}
\usepackage[utf8]{inputenc}
\usepackage[light]{draftcopy}
\usepackage[draft,nomargin,inline,marginclue]{fixme}

\newboolean{TEX4HT-HACKS}
\ifx\HCode\undefined
    \usepackage{mathptmx,
      courier
    }
    \usepackage[scaled=.90]{helvet}
    \setboolean{TEX4HT-HACKS}{false}
\else
    \setboolean{TEX4HT-HACKS}{true}
\fi

\floatstyle{plaintop}
\newfloat{listing}{tbp}{lol}[chapter]
\floatname{listing}{Listing}
\newcommand{\listoflistings}{\listof{listing}{Listings}}

\bibliographystyle{plainurl}

\renewcommand{\t}[1]{\texttt{#1}}
\renewcommand{\i}[1]{\textit{#1}}
\newcommand{\e}[1]{\emph{#1}}
\renewcommand{\b}[1]{\textbf{#1}}
\newcommand{\note}[1]{\paragraph{Note:} #1}
\newcommand{\TODOBUG}[2]{\fixme[inline]{(discussion on bug \##1) #2}}
\newcommand{\TODO}[1]{\fixme[inline]{#1}}

\definecolor{deepblue}{rgb}{0.0, 0.2, 0.7}
\definecolor{deeppurple}{rgb}{0.7, 0.0, 0.8}

\newboolean{ENABLE-ALL-OPTIONS}
\newboolean{ENABLE-KDEBUILD}

%%% Enable the below option if you'd like to see both sides of KDEBUILD conditionals shown in
%%% different colours. Disable it to either fully enable or fully disable KDEBUILD.
%%% Not compatible with HTML output.
\setboolean{ENABLE-ALL-OPTIONS}{false}

%%% Enable the below if you'd like to see KDEBUILD things.
\setboolean{ENABLE-KDEBUILD}{true}

\ifthenelse{\boolean{ENABLE-ALL-OPTIONS}\and\not\boolean{TEX4HT-HACKS}}
{
    \newcommand{\IFKDEBUILDELSE}[2]{{\def\mycolour{\color{deepblue}}\mycolour #1}{\def\mycolour{\color{deeppurple}}\mycolour #2}}
}{
    \ifthenelse{\boolean{ENABLE-KDEBUILD}}
    {
        \newcommand{\IFKDEBUILDELSE}[2]{#1}
    }{
        \newcommand{\IFKDEBUILDELSE}[2]{#2}
    }
}

\newenvironment{centertable}[1]%
{
  \begin{table}
    \ifx\mycolour\undefined\else\mycolour\fi
    \centering
    \caption{#1}
  }{
  \end{table}
}

\hypersetup{%
  urlcolor=black,
  colorlinks=true,
  citecolor=black,
  linkcolor=black,
  pdftitle={Package Manager Specification},
  pdfauthor={Stephen P. Bennet, Ciaran McCreesh},
  pdfcreator={pdfLaTeX and hyperref},
  pdfsubject={Defining a feature set for package managers in the
    Gentoo world},
  pdflang={en},
  pdfkeywords={Gentoo, package manager, specification, draft},
  pdfproducer={pdfLaTeX and hyperref},
}

\title{Package Manager Specification}
\author{Stephen P. Bennett\\\url{spb@exherbo.org}
\and Ciaran McCreesh\\\url{ciaran.mccreesh@googlemail.com}}

\begin{document}
\maketitle

\tableofcontents
\listofalgorithms
\listoflistings
\listoftables
\listoffixmes

\section*{Acknowledgements}

Thanks to Mike Kelly (package manager provided utilities, section~\ref{pkg-mgr-commands}),
Danny van Dyk (ebuild functions, section~\ref{ebuild-functions}), and
Petteri R\"aty (environment state, section~\ref{ebuild-env-state}) for contributions. Thanks also to
Mike Frysinger and Brian Harring for proof-reading and suggestions for fixes and/or clarification.

\section*{Copyright and Licence}

The bulk of this document is \textcopyright{} 2007 Stephen Bennett and Ciaran McCreesh. Contributions
are owned by their respective authors, and may have been changed substantially before inclusion.

This document is released under the Creative Commons Attribution-Share Alike 3.0 Licence. The full
text of this licence can be found at \url{http://creativecommons.org/licenses/by-sa/3.0/}.

\section*{Reporting Issues}

Issues (inaccuracies, wording problems, omissions etc.)\ in this document should be reported via
Gentoo Bugzilla using product \e{Gentoo Hosted Projects}, component \e{PMS/EAPI} and the default
assignee. There should be one bug per issue, and one issue per bug.

When reporting issues, remember that this document is not the appropriate place for pushing
through changes to the tree or the package manager, except where those changes are bugs.


% vim: set filetype=tex fileencoding=utf8 et tw=100 spell spelllang=en :

%%% Local Variables:
%%% mode: latex
%%% TeX-master: "pms"
%%% End:


\chapter{Introduction}

\section{Aims and Motivation}

This document aims to fully describe the format of an ebuild repository and the ebuilds therein, as
well as certain aspects of package manager behaviour required to support such a repository.

This document is \e{not} designed to be an introduction to ebuild development. Prior knowledge of
ebuild creation and an understanding of how the package management system works is assumed; certain
less familiar terms are explained in the Glossary in chapter~\ref{sec:glossary}.

This document does not specify any user or package manager configuration information.

\section{Rationale}

At present the only definition of what an ebuild can assume about its environment,
and the only definition of what is valid in an ebuild, is the source code of the latest Portage release
and a general consensus about which features are too new to assume availability. This has several
drawbacks: not only is it impossible to change any aspect of Portage behaviour without verifying
that nothing in the tree relies upon it, but if a new package manager should appear it becomes
impossible to fully support such an ill-defined standard.

This document aims to address both of these concerns by defining almost all aspects of what an
ebuild repository looks like, and how an ebuild is allowed to behave. Thus, both Portage and other
package managers can change aspects of their behaviour not defined here without worry of
incompatibilities with any particular repository.

\section{Reporting Issues}

Issues (inaccuracies, wording problems, omissions etc.)\ in this document should be reported via
Gentoo Bugzilla using product \e{Gentoo Hosted Projects}, component \e{PMS/EAPI} and the default
assignee. There should be one bug per issue, and one issue per bug.

Patches (in \t{git format-patch} form if possible) may be submitted either via Bugzilla or to the
\href{mailto:gentoo-pms@lists.gentoo.org}{\nolinkurl{gentoo-pms@lists.gentoo.org}} mailing list.
Patches will be reviewed by the PMS team, who will do one of the following:

\begin{compactitem}
\item Accept and apply the patch.
\item Explain why the patch cannot be applied as-is. The patch may then be updated and resubmitted
if appropriate.
\item Reject the patch outright.
\item Take special action merited by the individual circumstances.
\end{compactitem}

When reporting issues, remember that this document is not the appropriate place for pushing through
changes to the tree or the package manager, except where those changes are bugs.

If any issue cannot be resolved by the PMS team, it may be escalated to the Gentoo Council.

\section{Conventions}

Text in \t{teletype} is used for filenames or variable names. \i{Italic} text is used for terms
with a particular technical meaning in places where there may otherwise be ambiguity.

The term \i{package manager} is used throughout this document in a broad sense. Although some parts
of this document are only relevant to fully featured package managers, many items are equally
applicable to tools or other applications that interact with ebuilds or ebuild repositories.

\section{Acknowledgements}

Thanks to Mike Kelly (package manager provided utilities, section~\ref{sec:pkg-mgr-commands}),
Danny van Dyk (ebuild functions, section~\ref{sec:ebuild-functions}), David Leverton (various
sections), Petteri Räty (environment state, section~\ref{sec:ebuild-env-state}), Michał Górny
(various sections), Andreas K. Hüttel (stable use masking, section~\ref{sec:use-masking}),
Zac Medico (sub-slots, section~\ref{sec:mandatory-vars}) and James Le Cuirot (build dependencies,
section~\ref{sec:ebuild-env-vars}) for contributions. Thanks also to Mike Frysinger and
Brian Harring for proof-reading and suggestions for fixes and/or clarification.

% vim: set filetype=tex fileencoding=utf8 et tw=100 spell spelllang=en :

%%% Local Variables:
%%% mode: latex
%%% TeX-master: "pms"
%%% LaTeX-indent-level: 4
%%% LaTeX-item-indent: 0
%%% TeX-brace-indent-level: 4
%%% fill-column: 100
%%% End:


\chapter{Names and Versions}

\section{Restrictions upon Names}

No name may be empty. Package managers must not impose fixed upper boundaries upon the length of any
name.

\subsection{Category Names}
A category name may contain any of the characters [\t{A-Za-z0-9+\_.-}]. It must not begin with
a hyphen or a dot.

\note A hyphen is \i{not} required because of the \t{virtual} category. Usually, however, category
names will contain a hyphen.

\subsection{Package Names}
A package name may contain any of the characters [\t{A-Za-z0-9+\_-}]. It must not begin with a
hyphen, and must not end in a hyphen followed by one or more digits.

\note A package name does not include the category. The term \i{qualified package name} is used
where a \t{category/package} pair is meant.

\subsection{Slot Names}
\label{slot-names}
A slot name may contain any of the characters [\t{A-Za-z0-9+\_.-}]. It must not begin with a
hyphen or a dot.

\subsection{USE Flag Names}
A USE flag name may contain any of the characters [\t{A-Za-z0-9+\_@-}]. It must begin with an
alphanumeric character. Underscores should be considered reserved for \t{USE\_EXPAND}, as
described in section \ref{use-expand}.

\note The at-sign is required for \t{LINGUAS}.

\subsection{Repository Names}
\label{repository-names}
A repository name may contain any of the characters [\t{A-Za-z0-9\_-}]. It must not begin with a
hyphen.

\subsection{Keyword Names}
\label{keyword-names}
A keyword name may contain any of the characters [\t{A-Za-z0-9\_-}]. It must not begin with a
hyphen. In contexts where it makes sense to do so, a keyword name may be prefixed by
a tilde or a hyphen. In \t{KEYWORDS}, $-*$ is also acceptable as a keyword, to indicate that
a package will only work on listed targets.

A tilde prefixed keyword is, by convention, used to indicate a less stable package. It is generally
assumed that any user accepting keyword \t{\textasciitilde{}foo} will also accept \t{foo}.

\section{Version Specifications}
The package manager must not impose fixed limits upon the number of version components. No version
component represented by an integer may exceed the maximum value of a 32 bit signed two's complement
integer. All integers in version specifications must be non-negative.

Version specifications are given as ordered integer tuples, separated by (but not surrounded by)
decimal points, with any or none of the following suffixes in this order:
\begin{bulletlist}
\item One of the letters [\t{a-z}].
\item One of \t{\_alpha}, \t{\_beta}, \t{\_pre}, \t{\_rc}, and \t{\_p}, optionally followed by an
    integer. If the integer is omitted, it is assumed to be 0.
\item The string \t{-r} followed immediately by an integer (the ``revision number''). If this section
    is not present, it is assumed to be \t{-r0}.
\end{bulletlist}

\TODO{New portage allows multiple suffixes. There's bug 166522 open about this asking for the feature
    to be removed.}

\TODO{Portage has a really nasty -cvs thing that was designed to work around Portage's lack of
    ability to ignore versions it doesn't recognise. This should probably remain excluded and replaced
    by a proper solution in future EAPIs.}

\section{Version Comparison}

Version specifications are compared using strict integer comparison, moving left to right. \t{2.10}
is newer than \t{2.9}, which is newer than \t{1.500}. Any numeric value is newer than none, so
\t{1.0.0} is newer than \t{1.0} (and, by extension, \t{1.0.0\_alpha} is newer than \t{1.0}).
If the numeric parts are found to be equal, the suffixes above are considered, in the order listed.

Letter suffixes are compared alphabetically, with any letter being newer than no letter.

If letters are equal, the next part is considered. The ordering of this is complex: \t{\_alpha} is
older than \t{\_beta}, which is older than \t{\_pre}, which is older than \t{\_rc}, which is older
than no suffix, which is older than \t{\_p}. A suffix has an associated integer, which, if not
specified, defaults to \t{0}. If the two versions being compared have the same suffix, the
associated integer is compared.

If at this point the two versions are still equal, the revision number is compared. The revision
number has an optional integer suffix as per the previous part. If the revision numbers are equal,
so are the two versions.

\note Portage has slightly different behaviour where a version component begins with a zero. This
    behaviour is deliberately not specified here, as it is self-inconsistent and exists for legacy
    reasons only. The Gentoo tree should be fixed so as not to rely upon this behaviour, then it
    should be removed from future Portage releases.

\section{Uniqueness of versions}

No two packages in a given repository may have the same qualified package name and equal versions.
For example, a repository may not contain more than one of \t{foo-bar/baz-1.0.2},
\t{foo-bar/baz-1.0.2-r0} and \t{foo-bar/baz-1.000.2}.

% vim: set filetype=tex fileencoding=utf8 et tw=100 spell spelllang=en :


\chapter{Tree Layout}

This chapter defines the layout on-disk of an ebuild repository. In all cases below where a file or
directory is specified, a symlink to a file or directory is also valid. In this case, the package
manager should follow the operating system's semantics for symbolic links and not behave differently
from normal.

\section{Top Level}

An ebuild repository shall occupy one directory on disk, with the following subdirectories:
\begin{bulletlist}
\item One directory per category, whose name shall be the name of the category. The layout of
    these directories shall be as described in section \ref{category-dirs}.
\item A \t{profiles} directory, described in section \ref{profiles-dir}.
\item A \t{licenses} directory (optional), described in section \ref{licenses-dir}.
\item An \t{eclass} directory (optional), described in section \ref{eclass-dir}.
\item A \t{metadata} directory (optional), described in section \ref{metadata-dir}.
\item Other optional support files and directories (skeleton ebuilds or ChangeLogs,
    for example) may exist but are not covered by this specification. The package manager must
    ignore any of these files or directories that it does not recognise.

\end{bulletlist}

\section{Category Directories}
\label{category-dirs}

Each category provided by the repository (see also: the \t{profiles/categories} file, section
\ref{profiles-categories}) shall be contained in one directory, whose name shall be that of the
category. Each category directory shall contain:
\begin{bulletlist}
\item A \t{metadata.xml} file, as described in appendix \ref{metadata-xml}. Optional.
\item Zero or more package directories, one for each package in the category, as described in section
    \ref{package-dirs}. The name of the package directory shall be the corresponding package name.
\end{bulletlist}

Category directories may contain additional files, whose purpose is not covered by this
specification. Additional directories that are not for a package may \i{not} be present, to avoid
conflicts with package name directories; an exception is made for filesystem components whose name
starts with a dot, which the package manager must ignore.

\TODO{Explicitly ignore CVS too?}

\section{Package Directories}
\label{package-dirs}

A package directory contains the following:
\begin{bulletlist}
\item One or more ebuilds. These are as described in section \ref{ebuild-format} and others.
\item A \t{metadata.xml} file, as described in appendix \ref{metadata-xml}. Optional only for
    legacy support.
\item A \t{ChangeLog}, in a format determined by the provider of the respository. Optional.
\item A \t{Manifest} file, whose format is described in \cite{Glep44}.
\item A \t{files} directory, containing any support files needed by the ebuilds. Optional.
\end{bulletlist}

\section{The Profiles Directory}
\label{profiles-dir}

The profiles directory shall contain zero or more profile directories as described in section
\ref{profiles}, as well as the following files and directories. In any line-based file, lines
beginning with a \# character are treated as comments, whilst blank lines are ignored. All contents
of this directory, with the exception of \t{repo\_name}, are optional if the repository is not
intended to be stand-alone; if they are not present their contents are to be taken, where necessary,
from the master repository. Other files may exist, but may not be relied upon. The package manager
must ignore any files in this directory that it does not recognise.

\begin{description}
\item[arch.list] \label{arch.list} Contains a list, one entry per line, of permissible values for
    the \t{ARCH} variable, and hence permissible keywords for packages in this repository.
\item[categories] \label{profiles-categories} Contains a list, one entry per line, of categories
    provided by this repository.
\item[info\_pkgs] Contains a list, one entry per line, of qualified package names. Any package
    matching one of these is to be listed when a package manager displays a `system information'
    listing.
\item[info\_vars] Contains a list, one entry per line, of profile, configuration, and environment
    variables which are considered to be of interest. The value of each of these is to be included
    when the package manager displays a `system information' listing.
\item[package.mask] \label{profiles-package.mask}
    Contains a list, one entry per line, of (EAPI-0) dependency atoms. Any package
    version matching one of these is considered to be masked, and will not be installed regardless
    of profile unless it is unmasked by the user configuration.
\item[profiles.desc] Described below in section \ref{profiles.desc}.
\item[repo\_name] Contains, on a single line, the name of this repository. The repository name must
    conform to section \ref{repository-names}.
\item[thirdpartymirrors] Described below in section \ref{thirdpartymirrors}.
\item[use.desc] Contains descriptions of valid global USE flags for this repository. The format is
    described in section \ref{use.desc}.
\item[use.local.desc] Contains descriptions of valid local USE flags for this repository, along with
    the packages to which they apply. The format is as described in section \ref{use.desc}.
\item[desc/] This directory contains files analogous to \t{use.desc} for the various \t{USE\_EXPAND}
    variables. Each file in it is named \t{<varname>.desc}, where \t{<varname>} is the variable
    name, in lowercase, whose possible values the file describes. The format of each file is as for
    \t{use.desc}, described in section \ref{use.desc}. The \t{USE\_EXPAND} name is \i{not}
    included as a prefix here.
\item[updates/] This directory is described in section \ref{updates-dir}.
\end{description}

\subsection{The profiles.desc file}
\label{profiles.desc}
\t{profiles.desc} is a line-based file, with the standard commenting rules from section
\ref{profiles-dir}, containing a list of profiles that are valid for use, along with their
associated architecture and status. Each line has the format:
\begin{verbatim}
<keyword> <profile path> stable|dev
\end{verbatim}
Where \t{<keyword>} is the default keyword for the profile and the \t{ARCH} for which the profile is
valid, \t{<profile path>} is the (relative) path from the \t{profiles} directory to the profile in
question, and the third field is either \t{stable} or \t{dev}, depending upon whether the profile is
reckoned to be `stable' for normal use. Fields are whitespace-delimited. The last field is of most
use to QA scanning tools, which can display certain errors with reduced severity should they appear
in a `dev' profile.

\subsection{The thirdpartymirrors file}
\label{thirdpartymirrors}
\t{thirdpartymirrors} is another simple line-based file, describing the valid mirrors for use with
\t{mirror://} URIs in this repository, and the associated download locations. The format of each
line is:
\begin{verbatim}
<mirror name> <mirror 1> <mirror 2> ... <mirror n>
\end{verbatim}
Fields are whitespace-delimited. When parsing a URI of the form \t{mirror://name/filename}, the
\t{thirdpartymirrors} file is searched for a line whose first field is \t{name}. Then the download
URIs in the subsequent fields have \t{filename} appended to them to generate the URIs from which a
download is attempted.

\subsection{use.desc and related files}
\label{use.desc}
\t{use.desc} contains descriptions of every valid global USE flag for this repository. It is a
line-based file with the standard rules for comments and blank lines. The format of each line is:
\begin{verbatim}
<flagname> - <description>
\end{verbatim}

\t{use.local.desc} contains descriptions of every valid local USE flag---those that apply only to a
small number of packages, or that have different meanings for different packages. Its format is:
\begin{verbatim}
<category/package>:<flagname> - <description>
\end{verbatim}
Flags must be listed once for each package to which they apply, or if a flag is listed in both
\t{use.desc} and \t{use.local.desc}, it must be listed once for each package for which its meaning
differs from that described in \t{use.desc}.

\subsection{The updates directory}
\label{updates-dir}
The \t{updates} directory is used to inform the package manager that a package has moved categories,
names, or that a version has changed SLOT. It contains one file per quarter year, named
\t{[1-4]Q-[YYYY]} for the first to fourth quarter of a given year, for example \t{1Q-2004} or
\t{3Q-2006}. The format of each file is again line-based, with each line having one of the following
formats:
\begin{verbatim}
move <qpn1> <qpn2>
slotmove <atom> <slot1> <slot2>
\end{verbatim}
The first form, where \t{qpn1} and \t{qpn2} are \i{qualified package names}, instructs the package
manager that the package \t{qpn1} has changed name, category, or both, and is now called \t{qpn2}.

The second form instructs the package manager that any currently installed package version matching
\t{atom} whose \t{SLOT} is set to \t{slot1} should have it updated to \t{slot2}.


\section{The Licenses Directory}
\label{licenses-dir}

The \t{licenses} directory shall contain copies of the licenses used by packages in the
repository. Each file will be named according to the name used in the \t{LICENSE} variable as
described in section \ref{ebuild-var-LICENSE}, and will contain the complete text of the license in
human-readable form. Plain text format is strongly preferred but not required.

\section{The Eclass Directory}
\label{eclass-dir}

The \t{eclass} directory shall contain copies of the eclasses provided by this repository. The
format of these files is described in section \ref{eclasses}. It may also contain, in their own
directory, support files needed by these eclasses.

\section{The Metadata Directory}
\label{metadata-dir}

The \t{metadata} directory contains various repository-level metadata that is not contained in
\t{profiles/}. All contents are optional. In this standard only the \t{cache} subdirectory is
described; other contents are optional but may include security advisories, DTD files for the
various XML files used in the repository, and repository timestamps.

\subsection{The metadata cache}

The \t{metadata/cache} directory contains a cached form of all important ebuild metadata variables.
The cache directory, if it exists, contains (up to) one directory per category in the repository---
not all categories and packages must be contained in it. Each subdirectory contains one file per
package version, named \t{<package>-<version>}, in the following format:

Each cache file contains the textual values of various metadata keys, one per line, in the following
order. Other lines may be present following these; their meanings are not defined here.

\begin{enumerate}
\item Build-time dependencies (\t{DEPEND})
\item Run-time dependencies (\t{RDEPEND})
\item Slot (\t{SLOT})
\item Source tarball URIs (\t{SRC\_URI})
\item \t{RESTRICT}
\item Package homepage (\t{HOMEPAGE})
\item Package license (\t{LICENSE})
\item Package description (\t{DESCRIPTION})
\item Package keywords (\t{KEYWORDS})
\item Inherited eclasses (\t{INHERITED})
\item Use flags that this package respects (\t{IUSE})
\item No longer used; this line is to be ignored.
\item Post dependencies (\t{PDEPEND})
\item Old-style virtuals provided by this package (\t{PROVIDE})
\item The ebuild API version to which this package conforms (\t{EAPI})
\item Blank lines to pad the file to 22 lines long
\end{enumerate}

% vim: set filetype=tex fileencoding=utf8 et tw=100 spell spelllang=en :


\chapter{Profiles}
\label{sec:profiles}

\section{General principles}
Generally, a profile defines information specific to a certain `type' of system---it lies somewhere
between repository-level defaults and user configuration in that the information it contains is not
necessarily applicable to all machines, but is sufficiently general that it should not be left to
the user to configure it. Some parts of the profile can be overridden by user configuration, some
only by another profile.

The format of a profile is relatively simple. Each profile is a directory containing any number of
the files described in this chapter, and possibly inheriting another profile. The files themselves
follow a few basic conventions as regards inheritance and format; these are described in the next
section. It may also contain any number of subdirectories containing other profiles.

\section{Files that make up a profile}

\subsection{The parent file}
A profile may contain a \t{parent} file. Each line must contain a relative path to another profile
which will be considered as one of this profile's parents. Any settings from the parent are
inherited by this profile, and can be overridden by it. Precise rules for how settings are combined
with the parent profile vary between files, and are described below. Parents are handled depth
first, left to right, with duplicate parent paths being sourced for every time they are encountered.

It is illegal for a profile's parent tree to contain cycles. Package manager behaviour upon
encountering a cycle is undefined.

This file must not contain comments, blank lines or make use of line continuations.

\subsection{The eapi file}
\label{sec:profile-eapi}
A profile directory may contain an \t{eapi} file. This file, if it exists, must contain a single line
with the name of an EAPI. This specifies the EAPI to use when handling the directory in question; a
package manager must not attempt to use any profile using a directory which requires an EAPI it does
not support. If no \t{eapi} file is present, EAPI 0 shall be used. The EAPI is not inherited via the
\t{parent} file.

\subsection{deprecated}
If a profile contains a file named \t{deprecated}, it is treated as such. The first line of this
file should contain the path from the \t{profiles} directory of the repository to a valid profile
that is the recommended upgrade path from this profile. The remainder of the file can contain any
text, which may be displayed to users using this profile by the package manager. This file is not
inherited---profiles which inherit from a deprecated profile are \e{not} deprecated.

This file must not contain comments or make use of line continuations.

\subsection{make.defaults}
\t{make.defaults} is used to define defaults for various environment and configuration variables.
This file is unusual in that it is not combined at a file level with the parent---instead, each
variable is combined or overridden individually as described in section~\ref{sec:profile-variables}.

The file itself is a line-based key-value format. Each line contains a single \verb|VAR="value"|
entry, where the value must be double quoted. A variable name must start with one of \t{a-zA-Z}
and may contain \t{a-zA-Z0-9\_-}. Additional syntax, which is a small subset of
bash syntax, is allowed as follows:

\begin{compactitem}
\item Variables to the right of the equals sign in the form \t{\$\{foo\}} or \t{\$foo} are recognised and
  expanded from variables previously set in this or earlier \t{make.defaults} files.
\item One logical line may be continued over multiple physical lines by escaping the newline with a
  backslash. This is also permitted inside quoted strings.
\item Backslashes, except for line continuations, are not allowed.
\end{compactitem}

\subsection{virtuals}
\label{sec:profiles-virtuals}
The \t{virtuals} file defines default providers for ``old-style'' virtual packages. It is a simple
line-based file, with each line containing two whitespace-delimited tokens. The first is a virtual
package name (for example, \t{virtual/alsa}) and the second is a qualified package name. Blank lines
and those beginning with a \# character are ignored. When attempting to resolve a virtual name to a
concrete package, the specification defined in the active profile's \t{virtuals} list should be used if no
provider is already installed.

The \t{virtuals} file is inherited in the simplest manner: all entries from the parent profile are
loaded, then entries from the current profile. If a virtual package name appears in both, the entry
in the parent profile is discarded.

\subsection{Simple line-based files}
\label{sec:line-stacking}
These files are a simple one-item-per-line list, which is inherited in the following manner: the
parent profile's list is taken, and the current profile's list appended. If any line begins with a
hyphen, then any lines previous to it whose contents are equal to the remainder of that line are
removed from the list. Once again, blank lines and those beginning with a \# are discarded.

\subsection{packages}
The \t{packages} file is used to define the `system set' for this profile.
After the above rules for inheritance and comments are applied, its lines must take one of two
forms: a package dependency specification prefixed by \t{*} denotes that the atom forms part of the
system set. A package dependency specification on its own may also appear for legacy reasons, but
should be ignored when calculating the system set.

\subsection{packages.build}
The \t{packages.build} file is used by Gentoo's Catalyst tool to generate stage1 tarballs, and has
no relevance to the operation of a package manager. It is thus outside the scope of this document,
but is mentioned here for completeness.

\subsection{package.mask}
\t{package.mask} is used to prevent packages from being installed on a given profile. Each line
contains one package dependency specification; anything matching this specification will not be
installed unless unmasked by the user's configuration.

Note that the \t{-spec} syntax can be used to remove a mask in a parent profile, but not
necessarily a global mask (from \t{profiles/package.mask}, section~\ref{profiles-package.mask}).

\note Portage currently treats \t{profiles/package.mask} as being on the leftmost branch of the
    inherit tree when it comes to \t{-lines}. This behaviour may not be relied upon.

\subsection{package.provided}
\t{package.provided} is used to tell the package manager that a certain package version should be
considered to be provided by the system regardless of whether it is actually installed. Because it
has severe adverse effects on USE-based and slot-based dependencies, its use is strongly deprecated
and package manager support must be regarded as purely optional.

\subsection{package.use}
The \t{package.use} file may be used by the package manager to override the default USE flags specified
by \t{make.defaults} on a per package basis.  The format is to have a package dependency specification,
and then a space delimited list of USE flags to enable.  A USE flag in the form of \t{-flag} indicates
that the package should have the USE flag disabled.  The package dependency specification is limited to
the forms defined by the directory's EAPI.

\subsection{USE masking and forcing}
\label{sec:use-masking}
This section covers the four files \t{use.mask}, \t{use.force}, \t{package.use.mask} and
\t{package.use.force}. They are described together because they interact in a non-trivial manner.

Simply speaking, \t{use.mask} and \t{use.force} are used to say that a given USE flag must never or
always, respectively, be enabled when using this profile. \t{package.use.mask} and
\t{package.use.force} do the same thing on a per-package, or per-version, basis. The precise manner
in which they interact is less simple, and is best described in terms of the algorithm used to
determine whether a flag is masked for a given package version. This is described in Algorithm~\ref{alg:use-masking}.
\begin{algorithm}
\caption{USE masking logic} \label{alg:use-masking}
\begin{algorithmic}[1]
\STATE let masked = false
\FOR{each profile in the inheritance tree, depth first}
    \IF{\t{use.mask} contains \i{flag}}
        \STATE let masked = true
    \ELSIF{\t{use.mask} contains \i{-flag}}
        \STATE let masked = false
    \ENDIF
    \FOR{each $line$ in package.use.mask, in order, for which the spec matches $package$}
        \IF{$line$ contains \i{flag}}
            \STATE let masked = true
        \ELSIF{$line$ contains \i{-flag}}
            \STATE let masked = false
        \ENDIF
    \ENDFOR
\ENDFOR
\end{algorithmic}
\end{algorithm}

The logic for \t{use.force} and \t{package.use.force} is identical. If a flag is both masked and
forced, the mask is considered to take precedence.

\t{USE\_EXPAND} values may be forced or masked by using \t{expand\_name\_value}.

A package manager may treat \t{ARCH} values that are not the current architecture as being masked.

\section{Profile variables}
\label{sec:profile-variables}

This section documents variables that have special meaning, or special behaviour, when defined in a
profile's \t{make.defaults} file.

\subsection{Incremental Variables}
\i{Incremental} variables must stack between parent and child profiles in the following manner:
Beginning with the highest parent profile, tokenise the variable's value based on whitespace and
concatenate the lists. Then, for any token $T$ beginning with a hyphen, remove it and any previous
tokens whose value is equal to $T$ with the hyphen removed, or, if $T$ is equal to \t{-*}, remove
all previous values. Note that because of this treatment, the order of tokens in the final result is
arbitrary, not necessarily related to the order of tokens in any given profile. The following
variables must be treated in this fashion:
\begin{compactitem}
\item \t{USE}
\item \t{USE\_EXPAND}
\item \t{USE\_EXPAND\_HIDDEN}
\item \t{CONFIG\_PROTECT}
\item \t{CONFIG\_PROTECT\_MASK}
\end{compactitem}

If the package manager supports any EAPI listed in table~\ref{tab:profile-iuse-injection-table} as
using profile-defined \t{IUSE} injection, the following variables must also be treated
incrementally; otherwise, the following variables may or may not be treated incrementally:
\begin{compactitem}
\item \t{IUSE\_IMPLICIT}
\item \t{USE\_EXPAND\_IMPLICIT}
\item \t{USE\_EXPAND\_UNPREFIXED}
\end{compactitem}

Other variables, except where they affect only package-manager-specific functionality (such as
Portage's \t{FEATURES} variable), must not be treated incrementally---later definitions shall
completely override those in parent profiles.

\ChangeWhenAddingAnEAPI{5}
\begin{centertable}{Profile-defined \t{IUSE} injection for EAPIs} \label{tab:profile-iuse-injection-table}
    \begin{tabular}{ l l }
        \toprule
        \multicolumn{1}{c}{\textbf{EAPI}} &
        \multicolumn{1}{c}{\textbf{Supports profile-defined \t{IUSE} injection?}} \\
        \midrule
    \t{0} & No \\
    \t{1} & No \\
    \t{2} & No \\
    \t{3} & No \\
    \t{4} & No \\
    \t{5} & Yes \\
    \bottomrule
    \end{tabular}
\end{centertable}

\subsection{Specific variables and their meanings}
The following variables have specific meanings when set in profiles.
\begin{description}
\item[ARCH] The system's architecture. Must be a value listed in \t{profiles/arch.list}; see
    section~\ref{arch.list} for more information. Must be equal to the primary \t{KEYWORD} for this
    profile.
\item[CONFIG\_PROTECT, CONFIG\_PROTECT\_MASK] Contain whitespace-delimited lists used to control the
    configuration file protection. Described more fully in chapter~\ref{sec:config-protect}.
\item[USE] Defines the list of default USE flags for this profile. Flags may be added or removed by
    the user's configuration. \t{USE\_EXPAND} values must not be specified in this way.
\item[USE\_EXPAND] Defines a list of variables which are to be treated incrementally and whose
    contents are to be expanded into the USE variable as passed to ebuilds. See
    section~\ref{sec:use-iuse-handling} for details.
\item[USE\_EXPAND\_UNPREFIXED] Similar to \t{USE\_EXPAND}, but no prefix is used. If the repository
    contains any package using an EAPI supporting profile-defined \t{IUSE} injection (see
    table~\ref{tab:profile-iuse-injection-table}), this list must contain at least \t{ARCH}. See
    section~\ref{sec:use-iuse-handling} for details.
\item[USE\_EXPAND\_HIDDEN] Contains a (possibly empty) subset of names from \t{USE\_EXPAND} and
    \t{USE\_EXPAND\_UNPREFIXED}. The package manager may use this set as a hint to avoid displaying
    uninteresting or unhelpful information to an end user.
\item[USE\_EXPAND\_IMPLICIT, IUSE\_IMPLICIT] Used to inject implicit values into \t{IUSE}. See
    section~\ref{sec:use-iuse-handling} for details.
\end{description}

In addition, for EAPIs listed in table~\ref{tab:profile-iuse-injection-table} as supporting profile
defined \t{IUSE} injection, the variables named in \t{USE\_EXPAND} and \t{USE\_EXPAND\_UNPREFIXED}
have special handling as described in section~\ref{sec:use-iuse-handling}.

Any other variables set in \t{make.defaults} must be passed on into the ebuild environment as-is,
and are not required to be interpreted by the package manager.

% vim: set filetype=tex fileencoding=utf8 et tw=100 spell spelllang=en :

%%% Local Variables:
%%% mode: latex
%%% TeX-master: "pms"
%%% LaTeX-indent-level: 4
%%% LaTeX-item-indent: 0
%%% TeX-brace-indent-level: 4
%%% End:


% vim: set filetype=tex fileencoding=utf8 et tw=100 spell spelllang=en :

%%% Local Variables:
%%% mode: latex
%%% TeX-master: "pms"
%%% LaTeX-indent-level: 4
%%% LaTeX-item-indent: 0
%%% TeX-brace-indent-level: 4
%%% End:


\chapter{Old-Style Virtual Packages}
\label{old-virtuals}

Old-style virtuals are pseudo-packages---they can be depended upon or installed, but do not exist in
the ebuild repository.  An old-style virtual requires several things in the repository: at least one
ebuild must list the virtual in its \t{PROVIDE} variable, and there must be at least one entry in a
profiles \t{virtuals} file listing the default provider for each profile---see sections
\ref{ebuild-var-provide} and \ref{profiles-virtuals} for specifics on these two. Old-style virtuals
require special handling as regards dependencies; this is described below.

All old-style virtuals must use the category \t{virtual}. Not all packages using the \t{virtual}
category may be assumed to be old style virtuals.

\note A \i{new-style} virtual is simply an ebuild which install no files and use its dependency
strings to select providers. By convention, and to ease migration, these are also placed in the
\t{virtual} category.

\section{Dependencies on virtual packages}

When a dependency on a virtual package is encountered, it must be resolved into a real package
before it can be satisfied. There are two factors that affect this process: whether a package
providing the virtual is installed, and the \t{virtuals} file in the active profile (section
\ref{profiles-virtuals}). If a package is already installed which satisfies the virtual requirement
(via \t{PROVIDE}), then it should be used to satisfy the dependency. Otherwise, the profiles
\t{virtuals} file (section \ref{profiles-virtuals}) should be consulted to choose an appropriate
provider.

Dependencies on old style virtuals must not use any kind of version restriction.

Blocks on provided virtuals have special behaviour documented in section \ref{provided-blocks}.

% vim: set filetype=tex fileencoding=utf8 et tw=100 spell spelllang=en :

%%% Local Variables:
%%% mode: latex
%%% TeX-master: "pms"
%%% End:


\chapter{Ebuild File Format}
\label{ebuild-format}

% vim: set filetype=tex fileencoding=utf8 et tw=100 spell spelllang=en :


\chapter{Eclasses}
\label{sec:eclasses}

Eclasses serve to store common code that is used by more than one ebuild, which greatly aids
maintainability and reduces the tree size. However, due to metadata cache issues, care must be taken
in their use. In format they are similar to an ebuild, and indeed are sourced as part of any ebuild
using them. The interpreter is therefore the same, and the same requirements for being parseable
hold.

Eclasses must be located in the \t{eclass} directory in the top level of the repository---see
section~\ref{sec:eclass-dir}. Each eclass is a single file named \t{<name>.eclass}, where \t{<name>} is
the name of this eclass, used by \t{inherit} and \t{EXPORT_FUNCTIONS} among other places.

\section{The inherit Command}
\label{sec:inherit}

An ebuild wishing to make use of an eclass does so by using the \t{inherit} command in global scope.
This will cause the eclass to be sourced as part of the ebuild---any function or variable
definitions in the eclass will appear as part of the ebuild, with exceptions for certain metadata
variables, as described below.

The \t{inherit} command takes one or more parameters, which must be the names of eclasses (excluding
the \t{.eclass} suffix and the path). For each parameter, in order, the named eclass is sourced.

Eclasses may end up being sourced multiple times.

The \t{inherit} command must also ensure that:

\begin{compactitem}
\item The \t{ECLASS} variable is set to the name of the current eclass, when sourcing that eclass.
\item Once all inheriting has been done, the \t{INHERITED} metadata variable contains the name of
    every eclass used, separated  by whitespace.
\end{compactitem}

\section{Eclass-defined Metadata Keys}

The \t{IUSE}, \t{REQUIRED_USE}, \t{DEPEND}, \t{BDEPEND}, \t{RDEPEND} and \t{PDEPEND} variables
are handled specially when set by an eclass. They must be accumulated across eclasses, appending
the value set by each eclass to the resulting value after the previous one is loaded. Then the
eclass-defined value is appended to that defined by the ebuild. In the case of \t{RDEPEND}, this
is done after the implicit \t{RDEPEND} rules in section~\ref{sec:rdepend-depend} are applied.

\section{EXPORT_FUNCTIONS}

There is one command available in the eclass environment that is neither available nor meaningful
in ebuilds---\t{EXPORT_FUNCTIONS}\@. This can be used to alias ebuild phase functions from the
eclass so that an ebuild inherits a default definition whilst retaining the ability to override and
call the eclass-defined version from it. The use of it is best illustrated by an example; this is
given in listing~\ref{lst:export-functions} and is a snippet from a hypothetical \t{foo.eclass}.

\begin{listing}
\caption{\t{EXPORT_FUNCTIONS} example: \t{foo.eclass}} \label{lst:export-functions}
\begin{verbatim}
foo_src_compile()
{
    econf --enable-gerbil \
            $(use_enable fnord)
    emake gerbil || die "Couldn't make a gerbil"
    emake || die "emake failed"
}

EXPORT_FUNCTIONS src_compile
\end{verbatim}
\end{listing}

This example defines an eclass \t{src_compile} function and uses \t{EXPORT_FUNCTIONS} to alias
it. Then any ebuild that inherits \t{foo.eclass} will have a default \t{src_compile} defined, but
should the author wish to override it he can access the function in \t{foo.eclass} by calling
\t{foo_src_compile}.

\t{EXPORT_FUNCTIONS} must only be used on ebuild phase functions. The function that is aliased
must be named \t{eclassname_phasefunctionname}, where \t{eclassname} is the name of the eclass.

% vim: set filetype=tex fileencoding=utf8 et tw=100 spell spelllang=en :

%%% Local Variables:
%%% mode: latex
%%% TeX-master: "pms"
%%% LaTeX-indent-level: 4
%%% LaTeX-item-indent: 0
%%% TeX-brace-indent-level: 4
%%% fill-column: 100
%%% End:


\chapter{Ebuild-defined Variables}
\label{sec:ebuild-vars}

\note This section describes variables that may or must be defined by ebuilds. For
variables that are passed from the package manager to the ebuild, see section~\ref{sec:ebuild-env-vars}.

\section{Metadata invariance}
\label{sec:metadata-invariance}

All ebuild-defined variables discussed in this chapter must be defined independently of
any system, profile or tree dependent data, and must not vary depending upon the ebuild
phase. In particular, ebuild metadata can and will be generated on a different system from that upon
which the ebuild will be used, and the ebuild must generate identical metadata every time it
is used.

Globally defined ebuild variables without a special meaning must similarly not rely upon
variable data.

\section{Mandatory Ebuild-defined Variables}

All ebuilds must define at least the following variables:

\begin{description}
\item[DESCRIPTION] A short human-readable description of the package's purpose. May be defined by an
    eclass. Must not be empty.
\item[HOMEPAGE] The URI or URIs for a package's homepage, including protocols. May be defined by an
    eclass. See section~\ref{sec:dependencies} for full syntax.
\item[IUSE] The \t{USE} flags used by the ebuild. Any eclass that works with \t{USE} flags
    must also set \t{IUSE}, listing only the variables used by that eclass. The package manager is
    responsible for merging these values. See section~\ref{sec:use-iuse-handling} for discussion on
    which values must be listed this variable.

    \featurelabel{iuse-defaults}
    In EAPIs shown in table~\ref{tab:iuse-defaults-table} as supporting \t{IUSE} defaults, any use flag
    name in \t{IUSE} may be prefixed by at most one of a plus or a minus sign. If such a prefix is
    present, the package manager may use it as a suggestion as to the default value of the use flag
    if no other configuration overrides it.
\item[KEYWORDS] A whitespace separated list of keywords for the ebuild. Each token must be a
    valid keyword name, as per section~\ref{sec:keyword-names}. May include \t{-*}, which
    indicates that the package will only work on explicitly listed archs. May include \t{-arch},
    which indicates that the package will not work on the specified arch. May be empty, which
    indicates uncertain functionality on any architecture. May be defined in an eclass.
\item[LICENSE] The package's license. Each text token must correspond to a tree ``licenses/'' entry
    (see section~\ref{sec:licenses-dir}). See section~\ref{sec:dependencies} for full syntax.
    May be defined by an eclass. \label{ebuild-var-LICENSE}
\item[SLOT] The package's slot. Must be a valid slot name, as per section~\ref{sec:slot-names}. May
    be defined by an eclass. Must not be empty.
\item[SRC\_URI] A list of source URIs for the package. Valid protocols are \t{http://},
    \t{https://}, \t{ftp://} and \t{mirror://} (see section~\ref{sec:thirdpartymirrors} for mirror behaviour).
    Fetch restricted packages may include URL parts consisting of just a filename. See
    section~\ref{sec:dependencies} for full syntax.
\end{description}

If any of these variables are undefined, or if any of these variables are set to invalid values,
the package manager's behaviour is undefined; ideally, an error in one ebuild should not prevent
operations upon other ebuilds or packages.

\begin{centertable}{EAPIs supporting \t{IUSE} defaults} \label{tab:iuse-defaults-table}
\IFKDEBUILDELSE
{
    \begin{tabular}{ l l }
        \toprule
        \multicolumn{1}{c}{\textbf{EAPI}} &
        \multicolumn{1}{c}{\textbf{Supports \t{IUSE} defaults?}} \\
        \midrule
    \t{0} & No \\
    \t{1} & Yes \\
    \t{kdebuild-1} & Yes \\
    \t{2} & Yes \\
    \t{3} & Yes \\
    \bottomrule
    \end{tabular}
}
{
    \begin{tabular}{ l l }
        \toprule
        \multicolumn{1}{c}{\textbf{EAPI}} &
        \multicolumn{1}{c}{\textbf{Supports \t{IUSE} defaults?}} \\
        \midrule
    \t{0} & No \\
    \t{1} & Yes \\
    \t{2} & Yes \\
    \t{3} & Yes \\
    \bottomrule
    \end{tabular}
}
\end{centertable}

\section{Optional Ebuild-defined Variables}

Ebuilds may define any of the following variables:

\begin{description}
\item[DEPEND] See section~\ref{sec:dependencies}.
\item[EAPI] The EAPI. See below for defaults.
\item[PDEPEND] See section~\ref{sec:dependencies}.
\IFKDEBUILDELSE
{
    \featurelabel{provide}
    \item[PROVIDE] Zero or more qualified package names of any \e{old style}
        virtuals provided by this package. See section~\ref{sec:dependencies} for full syntax.  In EAPIs
        listed in table~\ref{tab:provide-table} as not supporting \t{PROVIDE}, ebuilds must not set this
        variable and the package manager must reject any ebuild that does so.
        \label{ebuild-var-provide}
}{
    \item[PROVIDE] Zero or more qualified package names of any \e{old style}
        virtuals provided by this package. See section~\ref{sec:dependencies} for full syntax.
        \label{ebuild-var-provide}
}
\item[RDEPEND] See section~\ref{sec:dependencies}. For some EAPIs, \t{RDEPEND} has special behaviour
    for its value if unset and when used with an eclass. See section~\ref{sec:rdepend-depend} for
    details.
\item[RESTRICT] Zero or more behaviour restrictions for this package. See section~\ref{sec:restrict}
    for value meanings and section~\ref{sec:dependencies} for full syntax.
\item[PROPERTIES] \featurelabel{properties}
    Zero or more properties for this package. See section~\ref{sec:properties}
    for value meanings and section~\ref{sec:dependencies} for full syntax. For EAPIs listed in
    table~\ref{tab:properties-table} as having optional support, ebuilds must not rely upon the
    package manager recognising or understanding this variable in any way.
\item[S] The path to the temporary build directory, used by \t{src\_compile}, \t{src\_install}
    etc. Defaults to \t{\$\{WORKDIR\}/\$\{P\}}.
\end{description}

\IFKDEBUILDELSE
{
    \begin{centertable}{EAPIs supporting \t{PROVIDE}} \label{tab:provide-table}
    \begin{tabular}{ l l }
        \toprule
        \multicolumn{1}{c}{\textbf{EAPI}} &
        \multicolumn{1}{c}{\textbf{Supports \t{PROVIDE}?}} \\
        \midrule
    \t{0} & Yes \\
    \t{1} & Yes \\
    \t{kdebuild-1} & No \\
    \t{2} & Yes \\
    \t{3} & Yes \\
    \bottomrule
    \end{tabular}
    \end{centertable}
}{
}

\begin{centertable}{EAPIs supporting \t{PROPERTIES}} \label{tab:properties-table}
\IFKDEBUILDELSE
{
    \begin{tabular}{ l l }
        \toprule
        \multicolumn{1}{c}{\textbf{EAPI}} &
        \multicolumn{1}{c}{\textbf{Supports \t{PROPERTIES}?}} \\
        \midrule
    \t{0} & Optionally \\
    \t{1} & Optionally \\
    \t{kdebuild-1} & Optionally \\
    \t{2} & Optionally \\
    \t{3} & Yes \\
    \bottomrule
    \end{tabular}
}
{
    \begin{tabular}{ l l }
        \toprule
        \multicolumn{1}{c}{\textbf{EAPI}} &
        \multicolumn{1}{c}{\textbf{Supports \t{PROPERTIES}?}} \\
        \midrule
    \t{0} & Optionally \\
    \t{1} & Optionally \\
    \t{2} & Optionally \\
    \t{3} & Yes \\
    \bottomrule
    \end{tabular}
}
\end{centertable}

\subsection{EAPI}
\label{sec:pre-source-eapi}

An empty or unset \t{EAPI} value is equivalent to \t{0}. Ebuilds must not assume that they will get
a particular one of these two values if they are expecting one of these two values.

The package manager must either pre-set the \t{EAPI} variable to \t{0} or ensure that it is unset
before sourcing the ebuild for metadata generation. When using the ebuild for other purposes, the
package manager must either pre-set \t{EAPI} to the value specified by the ebuild's metadata or
ensure that it is unset.

\IFKDEBUILDELSE
{
    \featurelabel{pre-source-eapi}
    When sourcing an ebuild with file extension \t{kdebuild-1}, the package manager must pre-set the
    \t{EAPI} variable to \t{kdebuild-1}; leaving it empty is not allowed. Ebuilds with this
    extension must not modify the \t{EAPI} variable.
}{
}

If any of these variables are set to invalid values, the package manager's behaviour is undefined;
ideally, an error in one ebuild should not prevent operations upon other ebuilds or packages.

\subsection{\t{RDEPEND} value}
\label{sec:rdepend-depend}

\featurelabel{rdepend-depend}
In EAPIs listed in table~\ref{tab:rdepend-depend-table} as having \t{RDEPEND=DEPEND}, if \t{RDEPEND}
is unset (but not if it is set to an empty string) in an ebuild, the package manager must set its
value to be equal to the value of \t{DEPEND}.

When dealing with eclasses, only values set in the ebuild itself are considered for this behaviour;
any \t{DEPEND} or \t{RDEPEND} set in an eclass does not change the implicit \t{RDEPEND=\$DEPEND} for
the ebuild portion, and any \t{DEPEND} value set in an eclass does not get added to \t{RDEPEND}.

\begin{centertable}{EAPIs with \t{RDEPEND=DEPEND} Default} \label{tab:rdepend-depend-table}
\IFKDEBUILDELSE
{
    \begin{tabular}{ l l }
        \toprule
        \multicolumn{1}{c}{\textbf{EAPI}} &
        \multicolumn{1}{c}{\textbf{\t{RDEPEND=DEPEND}?}} \\
        \midrule
    \t{0} & Yes \\
    \t{1} & Yes \\
    \t{kdebuild-1} & Yes \\
    \t{2} & Yes \\
    \t{3} & No \\
    \bottomrule
    \end{tabular}
}
{
    \begin{tabular}{ l l }
        \toprule
        \multicolumn{1}{c}{\textbf{EAPI}} &
        \multicolumn{1}{c}{\textbf{\t{RDEPEND=DEPEND}?}} \\
        \midrule
    \t{0} & Yes \\
    \t{1} & Yes \\
    \t{2} & Yes \\
    \t{3} & No \\
    \bottomrule
    \end{tabular}
}
\end{centertable}

\section{Magic Ebuild-defined Variables}

The following variables must be defined by \t{inherit}, and may be considered to be part
of the ebuild's metadata:

\begin{description}
\item[ECLASS] The current eclass, or unset if there is no current eclass. This is handled magically
    by \t{inherit} and must not be modified manually.
\item[INHERITED] List of inherited eclass names. Again, this is handled magically by \t{inherit}.
\end{description}

\note Thus, by extension of section~\ref{sec:metadata-invariance}, \t{inherit} may not be used
    conditionally, except upon constant conditions.

The following are special variables defined by the package manager for internal use and may or may
not be exported to the ebuild environment:

\begin{description}
\item[DEFINED\_PHASES] \featurelabel{defined-phases}
A space separated arbitrarily ordered list of phase names (e.g. \t{configure
setup unpack}) whose phase functions are defined by the ebuild or an eclass inherited by the ebuild.
If no phase functions are defined, a single hyphen is used instead of an empty string. For EAPIs
listed in table~\ref{tab:defined-phases-table} as having optional \t{DEFINED\_PHASES} support,
package managers may not rely upon the metadata cache having this variable defined, and must treat
an empty string as ``this information is not available''.
\end{description}

\note Thus, by extension of section~\ref{sec:metadata-invariance}, phase functions must not be defined
based upon any variant condition.

\begin{centertable}{EAPIs supporting \t{DEFINED\_PHASES}} \label{tab:defined-phases-table}
\IFKDEBUILDELSE
{
    \begin{tabular}{ l l }
        \toprule
        \multicolumn{1}{c}{\textbf{EAPI}} &
        \multicolumn{1}{c}{\textbf{Supports \t{DEFINED\_PHASES}?}} \\
        \midrule
    \t{0} & Optionally \\
    \t{1} & Optionally \\
    \t{kdebuild-1} & Optionally \\
    \t{2} & Optionally \\
    \t{3} & Yes \\
    \bottomrule
    \end{tabular}
}
{
    \begin{tabular}{ l l }
        \toprule
        \multicolumn{1}{c}{\textbf{EAPI}} &
        \multicolumn{1}{c}{\textbf{Supports \t{DEFINED\_PHASES}?}} \\
        \midrule
    \t{0} & Optionally \\
    \t{1} & Optionally \\
    \t{2} & Optionally \\
    \t{3} & Yes \\
    \bottomrule
    \end{tabular}
}
\end{centertable}

% vim: set filetype=tex fileencoding=utf8 et tw=100 spell spelllang=en :

%%% Local Variables:
%%% mode: latex
%%% TeX-master: "pms"
%%% End:


\chapter{Dependencies}
\label{dependencies}

\section{Dependency Classes}

There are three classes of dependencies supported by ebuilds:

\begin{bulletlist}
\item Build dependencies (\t{DEPEND}). These must be installed before the ebuild is installed.
\item Runtime dependencies (\t{RDEPEND}). These should usually be installed before the ebuild,
    but may be dropped to post dependencies where necessary to resolve cycles.
\item Post dependencies (\t{PDEPEND}). These should be installed at some point, usually after
    the ebuild if they are not already installed.
\end{bulletlist}

In addition, \t{SRC\_URI}, \t{PROVIDE} and \t{LICENSE} use the dependency specification format
to specify their values.

\note The term 'dependency specification' is perhaps not the best name, but it is the name
    most easily recognised by developers

\section{Dependency Specification Format}

The dependency specification format is a string containing zero or more of the following
items separated by whitespace:

\begin{bulletlist}
\item A package dependency specification (for \t{DEPEND} etc.), or a URI (for \t{SRC\_URI}),
    or a package name (for \t{PROVIDE}), or a license name (for \t{LICENSE}).
\item An all-of group, which consists of an open parenthesis, followed by whitespace,
    followed by (zero or more dependency items of any kind followed by whitespace), followed
    by a close parenthesis. More formally:
    \t{all-of ::= '(' whitespace (item whitespace)* ')'}.
\item An any-of group, which consists of the string \t{||}, followed by whitespace,
    followed by an open parenthesis, followed by whitespace, followed by (zero or more
    dependency items of any kind followed by whitespace), followed by a close parenthesis.
    More formally: \t{any-of ::= '||' whitespace '(' whitespace (item whitespace)* ')'}.
\item A use-conditional group, which consists of an optional exclamation mark, followed by
    a use flag name, followed by a question mark, followed by whitespace, followed by
    an open parenthesis, followed by whitespace, followed by (zero or more dependency items
    of any kind followed by whitespace), followed by a close parenthesis. More formally:
    \t{use-conditional ::= '!'? flag-name '?' whitespace '(' whitespace (item whitespace)* ')'}.
\end{bulletlist}

In particular, note that whitespace is not optional.

\subsection{Package Dependency Specifications}

In EAPI-0, a package dependency can be in one of the following base formats:

\begin{bulletlist}
\item A simple \t{category/package} name.
\item An operator, followed immediately by \t{category/package}, followed by a hyphen,
   followed by a version specification.
\end{bulletlist}

\TODOBUG{170161}{Should slot dependencies be included in EAPI-0?}

The following operators are available:

\begin{description}
\item[\t{<}] Strictly less than the specified version.
\item[\t{<=}] Less than or equal to the specified version.
\item[\t{=}] Exactly equal to the specified version. Special exception: if the version
    specified has an asterisk immediately following it, a string prefix comparison is
    used instead. When an asterisk is used, the specification must remain valid if the
    asterisk were removed. (An asterisk used with any other operator is illegal.)
\item[\t{\~}] Equal to the specified version, except the revision part of the matching
    package may be greater than the revision part of the specified version (\t{-r0} is
    assumed if no revision is explicitly stated).
\item[\t{>=}] Greater than or equal to the specified version.
\item[\t{>}] Strictly greater than the specified version.
\end{description}

If the operator is prefixed with an exclamation mark, the named dependency is a block
rather than a requirement---that is to say, the specified package must not be
installed, except with the following exceptions:

\begin{bulletlist}
\item Blocks on a package provided by the ebuild do not count. \label{provided-blocks}
\item Blocks on the ebuild itself do not count.
\end{bulletlist}

\subsection{All-of Dependency Specifications}

In an all-of group, all of the child elements must be matched.

\subsection{Use-conditional Dependency Specifications}

In a use-conditional group, if the associated use flag is enabled (or disabled if it has an
exclamation mark prefix), all of the child elements must be matched.

\subsection{Any-of Dependency Specifications}

Any-of dependency specifications only make sense for dependencies and licenses.

Any use-conditional group that is an immediate child of an any-of group, if not enabled (disabled
for an exclamation mark prefixed use flag name), is not considered a member of the any-of group
for dependency resolution purposes.

In an any-of group, at least one immediate child element must be matched. A blocker is
considered to be matched if it is not installed (or, for licenses, matched).

An empty any-of group counts as being installed (or, for licenses, matched).

% vim: set filetype=tex fileencoding=utf8 et tw=100 spell spelllang=en :



\chapter{Ebuild-defined Functions}
\label{ebuild-functions}

\section{List of Functions}
\label{functions}

The following is a list of functions that an ebuild, or eclass, may define, and which will be called
by the package manager as part of the build and/or install process. In all cases the package manager
must provide a default implementation of these functions; unless otherwise stated this must be a
no-op. Most functions must assume only that they have write access to the package's working
directory (the \t{WORKDIR} environment variable; see section \ref{env-var-WORKDIR}), and the
temporary directory \t{T}; exceptions are noted below. All functions may assume that they have read
access to all system libraries, binaries and configuration files that are accessible to normal
users.

Some functions may assume that their initial working directory is set to a particular location;
these are noted below. If no initial working directory is mandated, it may be set to anything and
the ebuild must not rely upon a particular location for it.

The environment for functions run outside of the build sequence (that is, \t{pkg\_config},
\t{pkg\_prerm} and \t{pkg\_postrm}) must be the environment used for the build of the package,
not the current configuration.

\subsection{pkg\_setup}
\label{pkg-setup-function}
The \t{pkg\_setup} function sets up the ebuild's environment for all following functions, before
the build process starts. Further, it checks whether any necessary prerequisites not covered
by the package manager, e.g. that certain kernel configuration options are fulfilled.

\t{pkg\_setup} must be run with full filesystem permissions, including the ability to add new users
and/or groups to the system.

\subsection{src\_unpack}
\label{src-unpack-function}

The \t{src\_unpack} function extracts all of the package's sources, applies patches and sets up the
package's build system for further use.

The initial working directory must be \t{WORKDIR}, and the default implementation used when
the ebuild lacks the \t{src\_unpack} function is:

\begin{lstlisting}
src_unpack() {
    if [ "${A}" != "" ]; then
        unpack ${A}
    fi
}
\end{lstlisting}

\subsection{src\_compile}
\label{src-compile-function}

The \t{src\_compile} function configures the package's build environment and builds the package.

The initial working directory must be \t{S} if that exists, falling back to \t{WORKDIR} otherwise.
The default implementation used when the ebuild lacks the \t{src\_compile} function is:

\begin{lstlisting}
src_compile() {
    if [ -x ./configure ]; then
        econf
    fi
    if [ -f Makefile ] || [ -f GNUmakefile ] || [ -f makefile ]; then
        emake || die "emake failed"
    fi
}
\end{lstlisting}

\subsection{src\_test}
\label{src-test-function}

The \t{src\_test} function runs unit tests for the newly built but not yet installed package as
provided.

The initial working directory must be \t{S} if that exists, falling back to \t{WORKDIR} otherwise.
The default implementation used when the ebuild lacks the \t{src\_test} function must, if tests are
enabled, run \t{make check} or \t{make test}, whichever exists, and abort if this fails.

\subsection{src\_install}
\label{src-install-function}

The \t{src\_install} function installs the package's content to a directory specified in
\t{\${D}}.

The initial working directory must be \t{S} if that exists, falling back to \t{WORKDIR} otherwise.
The default implementation used when the ebuild lacks the \t{src\_install} function is a no-op.

\subsection{pkg\_preinst}
\label{pkg-preinst-function}

The \t{pkg\_preinst} function performs any special tasks that are required immediately before
merging the package to the live filesystem. It must not write outside of the directories specified
by the \t{ROOT} and \t{D} environment variables.

\t{pkg\_preinst} must be run with full access to all files and directories below that specified by
the \t{ROOT} and \t{D} environment variables.

\subsection{pkg\_postinst}
\label{pkg-postinst-function}

The \t{pkg\_postinst} function performs any special tasks that are required immediately after
merging the package to the live filesystem. It must not write outside of the directory specified
in the \t{ROOT} environment variable.

\t{pkg\_postinst}, like, \t{pkg\_preinst}, must be run with full access to all files and directories
below that specified by the \t{ROOT} environment variable.

\subsection{pkg\_config}
\label{pkg-config-function}

The \t{pkg\_config} performs any custom steps required to configure a package after it has been
fully installed. It is the only ebuild function which may be interactive and prompt for user input.

\t{pkg\_config} must be run with full access to all files and directories inside of \t{ROOT}.

\subsection{pkg\_nofetch}
\label{pkg-nofetch-function}

The \t{pkg\_nofetch} function is run when the fetch phase of an fetch-restricted ebuild is run, and
the relevant source files are not available. It should direct the user to download all relevant
source files from their respective locations, with notes concerning licensing if applicable.

\t{pkg\_nofetch} should require no write access to any part of the filesystem.

\section{Call Order}

The call order for installing a package is:

\begin{bulletlist}
\item \t{pkg\_setup}
\item \t{src\_unpack}
\item \t{src\_compile}
\item \t{src\_test} (except if \t{RESTRICT=test})
\item \t{src\_install}
\item \t{pkg\_preinst}
\item \t{pkg\_postinst}
\end{bulletlist}

The call order for uninstalling a package is:

\begin{bulletlist}
\item \t{pkg\_prerm}
\item \t{pkg\_postrm}
\end{bulletlist}

The call order for reinstalling a package is:

\begin{bulletlist}
\item \t{pkg\_setup}
\item \t{src\_unpack}
\item \t{src\_compile}
\item \t{src\_test} (except if \t{RESTRICT=test})
\item \t{src\_install}
\item \t{pkg\_preinst}
\item \t{pkg\_prerm} for the package being replaced
\item \t{pkg\_postrm} for the package being replaced
\item \t{pkg\_postinst}
\end{bulletlist}

The call order for upgrading or downgrading a package is:

\begin{bulletlist}
\item \t{pkg\_setup}
\item \t{src\_unpack}
\item \t{src\_compile}
\item \t{src\_test} (except if \t{RESTRICT=test})
\item \t{src\_install}
\item \t{pkg\_preinst}
\item \t{pkg\_postinst}
\item \t{pkg\_prerm} for the package being replaced
\item \t{pkg\_postrm} for the package being replaced
\end{bulletlist}

The \t{pkg\_config} and \t{pkg\_nofetch} functions are not called in a normal sequence.

For installing binary packages, the \t{src} phases are not called.

When building binary packages that are not to be installed locally, the \t{pkg\_preinst}
and \t{pkg\_postinst} functions are not called.

% vim: set filetype=tex fileencoding=utf8 et tw=100 spell spelllang=en :


\chapter{The Ebuild Environment}

\section{Defined Variables}
\label{ebuild-env-vars}

\subsection{Globally Defined Variables}

The following variables are available, read-only, in all phases of the ebuild environment. They
should not be modified by ebuild code.
\begin{description}
\item[P] Package name and version, without the revision part. For example, \t{vim-7.0.174}.
\item[PN] Package name, for example \t{vim}.
\item[PV] Package version, for example \t{7.0.174}.
\item[PR] Package revision, or \t{r0} if none exists.
\item[PVR] Package version and revision, for example \t{7.0.174-r0} or \t{7.0.174-r1}.
\item[PF] Package name, version, and revision, for example \t{vim-7.0.174-r1}.
\item[A] All source files available for the package. Does not include any that are disabled because
    of USE conditionals. The value is calculated from the base names of each element of the
    \t{SRC\_URI} ebuild metadata variable.
\item[CATEGORY] The package's category, for example \t{app-editors}.
\item[FILESDIR] The full path to the package's files directory, used for small support files or
    patches. See section \ref{package-dirs}.
\item[WORKDIR] The full path to the ebuild's working directory, in which all build data should be
    contained.
\item[T] The full path to a temporary directory for use by the ebuild.



% vim: set filetype=tex fileencoding=utf8 et tw=100 spell spelllang=en :


\section{The State of Variables Between Functions}
\label{sec:ebuild-env-state}

Exported and default scope variables are saved between functions. A non-local variable set in a
function earlier in the call sequence must have its value preserved for later functions, including
functions executed as part of a later uninstall.

\note{\t{pkg_pretend} is \emph{not} part of the normal call sequence, and does not take part in
environment saving.}

Variables that were exported must remain exported in later functions; variables with default
visibility may retain default visibility or be exported. Variables with special meanings to the
package manager are excluded from this rule.

Global variables must only contain invariant values (see~\ref{sec:metadata-invariance}). If a global
variable's value is invariant, it may have the value that would be generated at any given point
in the build sequence.

This is demonstrated by code listing~\ref{lst:env-saving}.

\begin{listing}
\caption{Environment state between functions} \label{lst:env-saving}
\begin{verbatim}
GLOBAL_VARIABLE="a"

src_compile()
{
    GLOBAL_VARIABLE="b"
    DEFAULT_VARIABLE="c"
    export EXPORTED_VARIABLE="d"
    local LOCAL_VARIABLE="e"
}

src_install(){
    [[ ${GLOBAL_VARIABLE} == "a" ]] \
        || [[ ${GLOBAL_VARIABLE} == "b" ]] \
        || die "broken env saving for globals"

    [[ ${DEFAULT_VARIABLE} == "c" ]] \
        || die "broken env saving for default"

    [[ ${EXPORTED_VARIABLE} == "d" ]] \
        || die "broken env saving for exported"

    [[ $(printenv EXPORTED_VARIABLE ) == "d" ]] \
        || die "broken env saving for exported"

    [[ -z ${LOCAL_VARIABLE} ]] \
        || die "broken env saving for locals"
}
\end{verbatim}
\end{listing}

\section{The State of the System Between Functions}

For the sake of this section:
\nobreakpar
\begin{compactitem}
\item Variancy is any package manager action that modifies either \t{ROOT} or \t{/} in any way that
    isn't merely a simple addition of something that doesn't alter other packages. This includes
    any non-default call to any \t{pkg} phase function except \t{pkg_setup}, a merge of any package
    or an unmerge of any package.
\item As an exception, changes to \t{DISTDIR} do not count as variancy.
\item The \t{pkg_setup} function may be assumed not to introduce variancy. Thus, ebuilds must not
    perform variant actions in this phase.
\end{compactitem}

The following exclusivity and invariancy requirements are mandated:
\nobreakpar
\begin{compactitem}
\item No variancy shall be introduced at any point between a package's \t{pkg_setup} being started
    up to the point that that package is merged, except for any variancy introduced by that package.
\item There must be no variancy between a package's \t{pkg_setup} and a package's \t{pkg_postinst},
    except for any variancy introduced by that package.
\item Any non-default \t{pkg} phase function must be run exclusively.
\item Each phase function must be called at most once during the build process for any given
    package.
\end{compactitem}

% vim: set filetype=tex fileencoding=utf8 et tw=100 spell spelllang=en :

%%% Local Variables:
%%% mode: latex
%%% TeX-master: "pms"
%%% LaTeX-indent-level: 4
%%% LaTeX-item-indent: 0
%%% TeX-brace-indent-level: 4
%%% fill-column: 100
%%% End:


\chapter{Merging and Unmerging}

\note{In this chapter, \i{file} and \i{regular file} have their Unix meanings.}

\section{Overview}

The merge process merges the contents of the \t{D} directory onto the filesystem under \t{ROOT}\@.
This is not a straight copy; there are various subtleties which must be addressed.

The unmerge process removes an installed package's files. It is not covered in detail in this
specification.

\section{Directories}

Directories are merged recursively onto the filesystem. The method used to perform the merge is not
specified, so long as the end result is correct. In particular, merging a directory may alter or
remove the source directory under \t{D}.

Ebuilds must not attempt to merge a directory on top of any existing file that is not either a
directory or a symlink to a directory.

\subsection{Permissions}

The owner, group and mode (including set*id and sticky bits) of the directory must be preserved,
except as follows:

\begin{compactitem}
\item Any directory owned by the user used to perform the build must become owned by the root user.
\item Any directory whose group is the primary group of the user used to perform the build must have
    its group be that of the root user.
\end{compactitem}

On SELinux systems, the SELinux context must also be preserved. Other directory attributes, including
modification time, may be discarded.

\subsection{Empty directories}

Behaviour upon encountering an empty directory is undefined. Ebuilds must not attempt to install an
empty directory.

\section{Regular Files}

Regular files are merged onto the filesystem (but see the notes on configuration file protection,
below). The method used to perform the merge is not specified, so long as the end result is correct.
In particular, merging a regular file may alter or remove the source file under \t{D}.

Ebuilds must not attempt to merge a regular file on top of any existing file that is not either a
regular file or a symlink to a regular file.

\subsection{Permissions}

The owner, group and mode (including set*id and sticky bits) of the file must be preserved, except
as follows:

\begin{compactitem}
\item Any file owned by the user used to perform the build must become owned by the root user.
\item Any file whose group is the primary group of the user used to perform the build must have
    its group be that of the root user.
\item The package manager may reduce read and write permissions on executable files that have a
    set*id bit set.
\end{compactitem}

On SELinux systems, the SELinux context must also be preserved. Other
file attributes may be discarded.

\subsection{File modification times}

\featurelabel{mtime-preserve}
In EAPIs listed in table~\ref{tab:mtime-preserve}, the package manager
must preserve modification times of regular files. This includes files
being compressed before merging. Exceptions to this are files newly
created by the package manager and binary object files being stripped
of symbols.

When preserving, the seconds part of every regular file's mtime must
be preserved exactly. The sub-second part must either be set to zero,
or set to the greatest value supported by the operating system and
filesystem that is not greater than the sub-second part of the
original time.

For any given destination filesystem, the package manager must ensure
that for any two preserved files $a$, $b$ in that filesystem the
relation $\mbox{mtime}(a) \leq \mbox{mtime}(b)$ still holds, if it
held under the original image directory.

In other EAPIs, the behaviour with respect to file modification times
is undefined.

\ChangeWhenAddingAnEAPI{7}
\begin{centertable}{Preservation of file modification times (mtimes)}
    \label{tab:mtime-preserve}
    \begin{tabular}{ll}
      \toprule
      \multicolumn{1}{c}{\textbf{EAPI}} &
      \multicolumn{1}{c}{\textbf{mtimes preserved?}} \\
      \midrule
      0, 1, 2           & Undefined \\
      3, 4, 5, 6, 7     & Yes       \\
      \bottomrule
    \end{tabular}
\end{centertable}

\subsection{Configuration file protection}
\label{sec:config-protect}

The package manager must provide a means to prevent user configuration files from being
overwritten by any package updates. The profile variables \t{CONFIG_PROTECT} and
\t{CONFIG_PROTECT_MASK} (section~\ref{sec:profile-variables}) control the paths for which this
must be enforced.

In order to ensure interoperability with configuration update tools, the following scheme must be
used by all package managers when merging any regular file:

\begin{compactenum}
\item If the directory containing the file to be merged is not listed in \t{CONFIG_PROTECT}, and
     is not a subdirectory of any such directory, and if the file is not listed in \t{CONFIG_PROTECT},
     the file is merged normally.
\item If the directory containing the file to be merged is listed in \t{CONFIG_PROTECT_MASK}, or
    is a subdirectory of such a directory, or if the file is listed in \t{CONFIG_PROTECT_MASK},
    the file is merged normally.
\item If no existing file with the intended filename exists, or the existing file has identical
    content to the one being merged, the file is installed normally.
\item Otherwise, prepend the filename with \t{._cfg0000_}. If no file with the new name exists,
    then the file is merged with this name.
\item Otherwise, increment the number portion (to form \t{._cfg0001_<name>}) and repeat step 4.
    Continue this process until a usable filename is found.
\item If 9999 is reached in this way, behaviour is undefined.
\end{compactenum}

\section{Symlinks}

Symlinks are merged as symlinks onto the filesystem. The link destination for a merged link shall be
the same as the link destination for the link under \t{D}, except as noted below. The method used to
perform the merge is not specified, so long as the end result is correct; in particular, merging a
symlink may alter or remove the symlink under \t{D}.

Ebuilds must not attempt to merge a symlink on top of a directory.

\subsection{Rewriting}

Any absolute symlink whose link starts with \t{D} must be rewritten with the leading \t{D} removed.
The package manager should issue a notice when doing this.

\section{Hard Links}

A hard link may be merged either as a single file with links or as multiple independent files.

\section{Other Files}

Ebuilds must not attempt to install any other type of file (FIFOs, device nodes etc).

% vim: set filetype=tex fileencoding=utf8 et tw=100 spell spelllang=en :

%%% Local Variables:
%%% mode: latex
%%% TeX-master: "pms"
%%% LaTeX-indent-level: 4
%%% LaTeX-item-indent: 0
%%% TeX-brace-indent-level: 4
%%% fill-column: 100
%%% End:


\chapter{Glossary}
\label{glossary}

This section contains explanations of some of the terms used in this document whose meaning may not
be immediately obvious.

\begin{description}
\item[qualified package name] A package name along with its associated category. For example,
    \t{app-editors/vim} is a qualified package name.

\end{description}


% vim: set filetype=tex fileencoding=utf8 et tw=100 spell spelllang=en :


\appendix

\chapter{metadata.xml}
\label{sec:metadata-xml}

The \t{metadata.xml} file is used to contain extra package- or category-level information beyond
what is stored in ebuild metadata. Its exact format is strictly beyond the scope of this document,
and is described in the DTD file located at \url{http://www.gentoo.org/dtd/metadata.dtd}.

\chapter{Unspecified Items}

The following items are not specified by this document, and must not be relied upon by ebuilds.
This is, of course, an incomplete list---it covers only the things that the authors know have
been abused in the past.

\begin{compactitem}
\item The \t{FEATURES} variable. This is Portage specific.
\item Similarly, any \t{PORTAGE\_} variable not explicitly listed.
\item Any Portage configuration file.
\item The VDB (\t{/var/db/pkg}). Ebuilds must not access this or rely upon it existing or being
    in any particular format.
\item The \t{portageq} command. The \t{has\_version} and \t{best\_version} commands are
    available as functions.
\item The \t{emerge} command.
\item Binary packages.
\item The \t{PORTDIR\_OVERLAY} variable, and overlay behaviour in general.
\end{compactitem}

\chapter{Historical Curiosities}

The items described in this chapter are included for information only. They were deprecated or
abandoned long before \t{EAPI} was introduced. Ebuilds must not use these features, and package
managers should not be changed to support them.

\section{If-else use blocks}

Historically, Portage supported if-else use conditionals, as shown by
listing~\ref{lst:if-else-use-listing}. The block before the colon would be taken if the condition
was met, and the block after the colon would be taken if the condition was not met.

This feature was deprecated and removed from the tree long before the introduction of \t{EAPI}.

\begin{listing}
  \caption{If-else use blocks}\label{lst:if-else-use-listing}
  \verbatiminput{if-else-use.listing}
\end{listing}

\section{cvs Versions}

Portage has very crude support for CVS packages. The package \t{foo} could contain a file named
\t{foo-cvs.1.2.3.ebuild}. This version would order \i{higher} than any non-CVS version (including
\t{foo-2.ebuild}). This feature has not seen real world use and breaks versioned dependencies, so
it must not be used.

\IFKDEBUILDELSE
{
    The \t{scm} version rules specified in section~\ref{scm-versions} solve all of these issues.
}{
}

% vim: set filetype=tex fileencoding=utf8 et tw=100 spell spelllang=en :

%%% Local Variables:
%%% mode: latex
%%% TeX-master: "pms"
%%% LaTeX-indent-level: 4
%%% LaTeX-item-indent: 0
%%% TeX-brace-indent-level: 4
%%% End:


\bibliography{pms}

\end{document}

% vim: set filetype=tex fileencoding=utf8 et tw=100 spell spelllang=en :


%%% Local Variables:
%%% mode: latex
%%% TeX-master: t
%%% End:
