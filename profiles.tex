\chapter{Profiles}
\label{sec:profiles}

\section{General Principles}
Generally, a profile defines information specific to a certain `type' of system---it lies somewhere
between repository-level defaults and user configuration in that the information it contains is not
necessarily applicable to all machines, but is sufficiently general that it should not be left to
the user to configure it. Some parts of the profile can be overridden by user configuration, some
only by another profile.

The format of a profile is relatively simple. Each profile is a directory containing any number of
the files described in this chapter, and possibly inheriting another profile. The files themselves
follow a few basic conventions as regards inheritance and format; these are described in the next
section. It may also contain any number of subdirectories containing other profiles.

\section{Files That Make up a Profile}

\subsection{The parent file}
A profile may contain a \t{parent} file. Each line must contain a relative path to another profile
which will be considered as one of this profile's parents. Any settings from the parent are
inherited by this profile, and can be overridden by it. Precise rules for how settings are combined
with the parent profile vary between files, and are described below. Parents are handled depth
first, left to right, with duplicate parent paths being sourced for every time they are encountered.

It is illegal for a profile's parent tree to contain cycles. Package manager behaviour upon
encountering a cycle is undefined.

This file must not contain comments, blank lines or make use of line continuations.

\subsection{The eapi file}
\label{sec:profile-eapi}
A profile directory may contain an \t{eapi} file. This file, if it exists, must contain a single
line with the name of an EAPI\@. This specifies the EAPI to use when handling the directory in
question; a package manager must not attempt to use any profile using a directory which requires an
EAPI it does not support. If no \t{eapi} file is present, EAPI 0 shall be used. The EAPI is neither
inherited via the \t{parent} file nor in subdirectories.

\subsection{deprecated}
If a profile contains a file named \t{deprecated}, it is treated as such. The first line of this
file should contain the path from the \t{profiles} directory of the repository to a valid profile
that is the recommended upgrade path from this profile. The remainder of the file can contain any
text, which may be displayed to users using this profile by the package manager. This file is not
inherited---profiles which inherit from a deprecated profile are \e{not} deprecated.

This file must not contain comments or make use of line continuations.

\subsection{make.defaults}
\t{make.defaults} is used to define defaults for various environment and configuration variables.
This file is unusual in that it is not combined at a file level with the parent---instead, each
variable is combined or overridden individually as described in section~\ref{sec:profile-variables}.

The file itself is a line-based key-value format. Each line contains a single \verb|VAR="value"|
entry, where the value must be double quoted. A variable name must start with one of \t{a-zA-Z}
and may contain \t{a-zA-Z0-9_} only. Additional syntax, which is a small subset of
bash syntax, is allowed as follows:

\begin{compactitem}
\item Variables to the right of the equals sign in the form \t{\$\{foo\}} or \t{\$foo} are recognised and
  expanded from variables previously set in this or earlier \t{make.defaults} files.
\item One logical line may be continued over multiple physical lines by escaping the newline with a
  backslash. A quoted string may be continued over multiple physical lines by either a simple
  newline or a backslash-escaped newline.
\item Backslashes, except for line continuations, are not allowed.
\end{compactitem}

\subsection{Simple line-based files}
\label{sec:line-stacking}
These files are a simple one-item-per-line list, which is inherited in the following manner: the
parent profile's list is taken, and the current profile's list appended. If any line begins with a
hyphen, then any lines previous to it whose contents are equal to the remainder of that line are
removed from the list. Once again, blank lines and those beginning with a \# are discarded.

\featurelabel{profile-file-dirs} In EAPIs listed in table~\ref{tab:profile-file-dirs} as supporting
directories for profile files, any of the files \t{package.mask}, \t{package.use}, \t{use.*} and
\t{package.use.*} mentioned below can be a directory instead of a regular file. Files contained
in that directory, unless their name begins with a dot, will be concatenated in order of their
filename in the POSIX locale and the result will be processed as if it were a single file.
Any subdirectories will be ignored.

\ChangeWhenAddingAnEAPI{7}
\begin{centertable}{EAPIs supporting directories for profile files}
    \label{tab:profile-file-dirs}
    \begin{tabular}{ll}
      \toprule
      \multicolumn{1}{c}{\textbf{EAPI}} &
      \multicolumn{1}{c}{\textbf{Supports directories for profile files?}} \\
      \midrule
      0, 1, 2, 3, 4, 5, 6 & No  \\
      7                   & Yes \\
      \bottomrule
    \end{tabular}
\end{centertable}

\subsection{packages}
The \t{packages} file is used to define the `system set' for this profile.
After the above rules for inheritance and comments are applied, its lines must take one of two
forms: a package dependency specification prefixed by \t{*} denotes that it forms part of the
system set. A package dependency specification on its own may also appear for legacy reasons, but
should be ignored when calculating the system set.

\subsection{packages.build}
The \t{packages.build} file is used by Gentoo's Catalyst tool to generate stage1 tarballs, and has
no relevance to the operation of a package manager. It is thus outside the scope of this document,
but is mentioned here for completeness.

\subsection{package.mask}
\t{package.mask} is used to prevent packages from being installed on a given profile. Each line
contains one package dependency specification; anything matching this specification will not be
installed unless unmasked by the user's configuration. In some EAPIs, \t{package.mask} can be a
directory instead of a regular file as per section~\ref{sec:line-stacking}.

Note that the \t{-spec} syntax can be used to remove a mask in a parent profile, but not
necessarily a global mask (from \t{profiles/package.mask}, section~\ref{profiles-package.mask}).

\note{Portage currently treats \t{profiles/package.mask} as being on the leftmost branch of the
inherit tree when it comes to \t{-lines}. This behaviour may not be relied upon.}

\subsection{package.provided}
\t{package.provided} is used to tell the package manager that a certain package version should be
considered to be provided by the system regardless of whether it is actually installed. Because it
has severe adverse effects on USE-based and slot-based dependencies, its use is strongly deprecated
and package manager support must be regarded as purely optional.

\subsection{package.use}
The \t{package.use} file may be used by the package manager to override the default USE flags
specified by \t{make.defaults} on a per package basis. The format is to have a package dependency
specification, and then a space delimited list of USE flags to enable. A USE flag in the form of
\t{-flag} indicates that the package should have the USE flag disabled. The package dependency
specification is limited to the forms defined by the directory's EAPI\@. In some EAPIs,
\t{package.use} can be a directory instead of a regular file as per section~\ref{sec:line-stacking}.

\subsection{USE masking and forcing}
\label{sec:use-masking}
This section covers the eight files \t{use.mask}, \t{use.force}, \t{use.stable.mask},
\t{use.stable.force}, \t{package.use.mask}, \t{package.use.force}, \t{package.use.stable.mask},
and \t{package.use.\allowbreak stable.force}. They are described together because they interact in
a non-trivial manner. In some EAPIs, these files can be directories instead of regular files as per
section~\ref{sec:line-stacking}.

Simply speaking, \t{use.mask} and \t{use.force} are used to say that a given USE flag must never or
always, respectively, be enabled when using this profile. \t{package.use.mask} and
\t{package.use.force} do the same thing on a per-package, or per-version, basis.

\featurelabel{stablemask}
In profile directories with an EAPI supporting stable masking, as listed in
table~\ref{tab:profile-stablemask}, the same is true for \t{use.stable.mask}, \t{use.stable.force},
\t{package.use.stable.mask} and \t{package.use.\allowbreak stable.force}. These files, however,
only act on packages that are merged due to a stable keyword in the sense of
subsection~\ref{sec:keywords}. Thus, these files can be used to restrict the feature set deemed
stable in a package.

\ChangeWhenAddingAnEAPI{7}
\begin{centertable}{Profile directory support for masking/forcing use flags in stable versions only}
    \label{tab:profile-stablemask}
    \begin{tabular}{ll}
      \toprule
      \multicolumn{1}{c}{\textbf{EAPI}} &
      \multicolumn{1}{c}{\textbf{Supports masking/forcing use flags in stable versions?}} \\
      \midrule
      0, 1, 2, 3, 4     & No  \\
      5, 6, 7           & Yes \\
      \bottomrule
    \end{tabular}
\end{centertable}

The precise manner in which the eight files interact is less simple, and is best described in terms
of the algorithm used to determine whether a flag is masked for a given package version. This is
described in algorithm~\ref{alg:use-masking}.
\begin{algorithm}
\caption{\t{USE} masking logic} \label{alg:use-masking}
\begin{algorithmic}[1]
\STATE let masked = false
\FOR{each profile in the inheritance tree, depth first}
    \IF{\t{use.mask} contains \i{flag}}
        \STATE let masked = true
    \ELSIF{\t{use.mask} contains \i{-flag}}
        \STATE let masked = false
    \ENDIF
    \IF{stable keyword in use}
        \IF{\t{use.stable.mask} contains \i{flag}}
            \STATE let masked = true
        \ELSIF{\t{use.stable.mask} contains \i{-flag}}
            \STATE let masked = false
        \ENDIF
    \ENDIF
    \FOR{each $line$ in package.use.mask, in order, for which the spec matches $package$}
        \IF{$line$ contains \i{flag}}
            \STATE let masked = true
        \ELSIF{$line$ contains \i{-flag}}
            \STATE let masked = false
        \ENDIF
    \ENDFOR
    \IF{stable keyword in use}
        \FOR{each $line$ in package.use.stable.mask, in order, for which the spec matches $package$}
            \IF{$line$ contains \i{flag}}
                \STATE let masked = true
            \ELSIF{$line$ contains \i{-flag}}
                \STATE let masked = false
            \ENDIF
        \ENDFOR
    \ENDIF
\ENDFOR
\end{algorithmic}
\end{algorithm}

Stable restrictions (``stable keyword in use'' in algorithm~\ref{alg:use-masking}) are applied
exactly if replacing in \t{KEYWORDS} all stable keywords by the corresponding tilde prefixed
keywords (see subsection~\ref{sec:keywords}) would result in the package installation being
prevented due to the \t{KEYWORDS} setting.

The logic for \t{use.force}, \t{use.stable.force}, \t{package.use.force}, and
\t{package.use.\allowbreak stable.force} is identical. If a flag is both masked and forced, the
mask is considered to take precedence.

\t{USE_EXPAND} values may be forced or masked by using \t{expand_name_value}.

A package manager may treat \t{ARCH} values that are not the current architecture as being masked.

\section{Profile variables}
\label{sec:profile-variables}

This section documents variables that have special meaning, or special behaviour, when defined in a
profile's \t{make.defaults} file.

\subsection{Incremental Variables}
\i{Incremental} variables must stack between parent and child profiles in the following manner:
Beginning with the highest parent profile, tokenise the variable's value based on whitespace and
concatenate the lists. Then, for any token $T$ beginning with a hyphen, remove it and any previous
tokens whose value is equal to $T$ with the hyphen removed, or, if $T$ is equal to \t{-*}, remove
all previous values. Note that because of this treatment, the order of tokens in the final result is
arbitrary, not necessarily related to the order of tokens in any given profile. The following
variables must be treated in this fashion:
\begin{compactitem}
\item \t{USE}
\item \t{USE\_EXPAND}
\item \t{USE\_EXPAND\_HIDDEN}
\item \t{CONFIG\_PROTECT}
\item \t{CONFIG\_PROTECT\_MASK}
\end{compactitem}

If the package manager supports any EAPI listed in table~\ref{tab:profile-iuse-injection-table} as
using profile-defined \t{IUSE} injection, the following variables must also be treated
incrementally; otherwise, the following variables may or may not be treated incrementally:
\begin{compactitem}
\item \t{IUSE\_IMPLICIT}
\item \t{USE\_EXPAND\_IMPLICIT}
\item \t{USE\_EXPAND\_UNPREFIXED}
\end{compactitem}

Other variables, except where they affect only package-manager-specific functionality (such as
Portage's \t{FEATURES} variable), must not be treated incrementally---later definitions shall
completely override those in parent profiles.

\ChangeWhenAddingAnEAPI{5}
\begin{centertable}{Profile-defined \t{IUSE} injection for EAPIs} \label{tab:profile-iuse-injection-table}
    \begin{tabular}{ l l l }
        \toprule
        \multicolumn{1}{c}{\textbf{EAPI}} &
        \multicolumn{1}{c}{\textbf{Supports profile-defined \t{IUSE} injection?}} \\
        \midrule
    \t{0} & No \\
    \t{1} & No \\
    \t{2} & No \\
    \t{3} & No \\
    \t{4} & No \\
    \t{5} & Yes \\
    \bottomrule
    \end{tabular}
\end{centertable}

\subsection{Specific variables and their meanings}
The following variables have specific meanings when set in profiles.
\begin{description}
\item[ARCH] The system's architecture. Must be a value listed in \t{profiles/arch.list}; see
    section~\ref{arch.list} for more information. Must be equal to the primary \t{KEYWORD} for this
    profile.
\item[CONFIG\_PROTECT, CONFIG\_PROTECT\_MASK] Contain whitespace-delimited lists used to control the
    configuration file protection. Described more fully in chapter~\ref{sec:config-protect}.
\item[USE] Defines the list of default USE flags for this profile. Flags may be added or removed by
    the user's configuration. \t{USE\_EXPAND} values must not be specified in this way.
\item[USE\_EXPAND] Defines a list of variables which are to be treated incrementally and whose
    contents are to be expanded into the USE variable as passed to ebuilds. See
    section~\ref{sec:use-iuse-handling} for details.
\item[USE\_EXPAND\_UNPREFIXED] Similar to \t{USE\_EXPAND}, but no prefix is used. If the repository
    contains any package using an EAPI supporting profile-defined \t{IUSE} injection (see
    table~\ref{tab:profile-iuse-injection-table}), this list must contain at least \t{ARCH}. See
    section~\ref{sec:use-iuse-handling} for details.
\item[USE\_EXPAND\_HIDDEN] Contains a (possibly empty) subset of names from \t{USE\_EXPAND} and
    \t{USE\_EXPAND\_UNPREFIXED}. The package manager may use this set as a hint to avoid displaying
    uninteresting or unhelpful information to an end user.
\item[USE\_EXPAND\_IMPLICIT, IUSE\_IMPLICIT] Used to inject implicit values into \t{IUSE}. See
    section~\ref{sec:use-iuse-handling} for details.
\end{description}

In addition, for EAPIs listed in table~\ref{tab:profile-iuse-injection-table} as supporting profile
defined \t{IUSE} injection, the variables named in \t{USE\_EXPAND} and \t{USE\_EXPAND\_UNPREFIXED}
have special handling as described in section~\ref{sec:use-iuse-handling}.

Any other variables set in \t{make.defaults} must be passed on into the ebuild environment as-is,
and are not required to be interpreted by the package manager.

% vim: set filetype=tex fileencoding=utf8 et tw=100 spell spelllang=en :

%%% Local Variables:
%%% mode: latex
%%% TeX-master: "pms"
%%% LaTeX-indent-level: 4
%%% LaTeX-item-indent: 0
%%% TeX-brace-indent-level: 4
%%% End:


% vim: set filetype=tex fileencoding=utf8 et tw=100 spell spelllang=en :

%%% Local Variables:
%%% mode: latex
%%% TeX-master: "pms"
%%% LaTeX-indent-level: 4
%%% LaTeX-item-indent: 0
%%% TeX-brace-indent-level: 4
%%% fill-column: 100
%%% End:
