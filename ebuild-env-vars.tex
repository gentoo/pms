\section{Defined Variables}
\label{ebuild-env-vars}

\subsection{Globally Defined Variables}

The following variables are available, read-only, in all phases of the ebuild environment. They
should not be modified by ebuild code.
\begin{description}
\item[P] Package name and version, without the revision part. For example, \t{vim-7.0.174}.
\item[PN] Package name, for example \t{vim}.
\item[PV] Package version, for example \t{7.0.174}.
\item[PR] Package revision, or \t{r0} if none exists.
\item[PVR] Package version and revision, for example \t{7.0.174-r0} or \t{7.0.174-r1}.
\item[PF] Package name, version, and revision, for example \t{vim-7.0.174-r1}.
\item[A] All source files available for the package. Does not include any that are disabled because
    of USE conditionals. The value is calculated from the base names of each element of the
    \t{SRC\_URI} ebuild metadata variable.
\item[CATEGORY] The package's category, for example \t{app-editors}.
\item[FILESDIR] The full path to the package's files directory, used for small support files or
    patches. See section \ref{package-dirs}.
\item[WORKDIR] The full path to the ebuild's working directory, in which all build data should be
    contained.
\item[T] The full path to a temporary directory for use by the ebuild.



% vim: set filetype=tex fileencoding=utf8 et tw=100 spell spelllang=en :
